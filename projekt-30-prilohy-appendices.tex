% Tento soubor nahraďte vlastním souborem s přílohami (nadpisy níže jsou pouze pro příklad)

% Pro kompilaci po částech (viz projekt.tex), nutno odkomentovat a upravit
%\documentclass[../projekt.tex]{subfiles}
%\begin{document}

% Umístění obsahu paměťového média do příloh je vhodné konzultovat s vedoucím
%\chapter{Obsah přiloženého paměťového média}

%\chapter{Manuál}

%\chapter{Konfigurační soubor}

%\chapter{RelaxNG Schéma konfiguračního souboru}

%\chapter{Plakát}
\chapter{Obsah připojeného úložiště média}

\begin{itemize}[leftmargin=2cm]
    
    \item[\texttt{application}] Adresář obsahující firmware digitálního záznamníku včetně vygenerované HTML a Doxygen dokumentace.

    \item[\texttt{hardware}] Adresář obsahující projekt expanzní desky, obsahuje schémata zapojení, návrhy desek plošných spojů (PCB), BOM a také kompletní schéma v PDF.

    \item[\texttt{logs}] Složka pro ukládání záznamů z provozu zařízení, například zaznamenaná data ze senzorů.

    \item[\texttt{power\_consumption}] Data z analýzy spotřeby energie změřených pomocí Power Profiler Kit II od společnosti Nordic Semiconductors.

    \item[\texttt{tests}] Testovací skript včetně testovacích souborů a výstupní soubory ze statické analýzy kódu, ke které byl využit nástroj PC-lint Plus.

    \item[\texttt{thesis}] Obsahuje samotný text technické zprávě k bakalářské práci, včetně obrázků a digramů.

    \item[\texttt{README.md}] Úvodní soubor s popisem projektu, jak jej nainstalovat a používat.
\end{itemize}


% Pro kompilaci po částech (viz projekt.tex) nutno odkomentovat
%\end{document}
