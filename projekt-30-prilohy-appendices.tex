% Tento soubor nahraďte vlastním souborem s přílohami (nadpisy níže jsou pouze pro příklad)

% Pro kompilaci po částech (viz projekt.tex), nutno odkomentovat a upravit
%\documentclass[../projekt.tex]{subfiles}
%\begin{document}

% Umístění obsahu paměťového média do příloh je vhodné konzultovat s vedoucím
\chapter{Obsah přiloženého paměťového média}


\begin{itemize}
    \item \texttt{application/} -- adresář obsahující firmware digitálního záznamníku, včetně vygenerované HTML a Doxygen dokumentace.
    \begin{itemize}
        \item \texttt{src/} -- adresář obsahující zdrojové soubory firmware digitálního záznamníku.
        \item \texttt{include/} -- adresář obsahující hlavičkové soubory firmware digitálního záznamníku.
        \item \texttt{doc/} -- adresář obsahující LaTeX a HTML Doxygen dokumentaci firmware digitálního záznamníku.
    \end{itemize}
    \item \texttt{hardware/} -- adresář s projektem expanzní desky: schématy zapojení, návrhem PCB, seznamem součástek (BOM) a schématem v PDF.

    \begin{itemize}
        \item \texttt{design/} -- adresář obsahující KiCAD projekt expanzní desky.
        \item \texttt{BOM.txt} -- adresář obsahující hlavičkové soubory firmware digitálního záznamníku.
    \end{itemize}
    
    \item \texttt{logs/} -- složka pro ukládání záznamů z provozu zařízení, například dat ze senzorů.
    
    \item \texttt{power\_consumption/} -- data z analýzy spotřeby energie, změřená pomocí Power Profiler Kit II od společnosti Nordic Semiconductors.
    
    \item \texttt{tests/} -- testovací skript, testovací soubory a výstupy ze statické analýzy kódu pomocí nástroje PC-lint Plus.

    \item \texttt{logs/} -- složka obsahující obrázky k README.

    \item \texttt{thesis/} -- adresář obsahující text technické zprávy k bakalářské práci, včetně obrázků a diagramů.
    
    \item \texttt{README.md} -- úvodní soubor s popisem projektu, instalace a způsobem použití.
\end{itemize}

% Pro kompilaci po částech (viz projekt.tex) nutno odkomentovat
%\end{document}
