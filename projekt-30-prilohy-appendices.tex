% Tento soubor nahraďte vlastním souborem s přílohami (nadpisy níže jsou pouze pro příklad)

% Pro kompilaci po částech (viz projekt.tex), nutno odkomentovat a upravit
%\documentclass[../projekt.tex]{subfiles}
%\begin{document}

% Umístění obsahu paměťového média do příloh je vhodné konzultovat s vedoucím
\chapter{Obsah přiloženého paměťového média}

\begin{itemize}
    \item \texttt{application/} -- adresář obsahující firmware digitálního záznamníku, včetně vygenerované HTML a Doxygen dokumentace
    \begin{itemize}
        \item \texttt{src/} -- adresář obsahující zdrojové soubory firmware digitálního záznamníku
        \item \texttt{include/} -- adresář obsahující hlavičkové soubory firmware digitálního záznamníku
        \item \texttt{doc/} -- adresář obsahující LaTeX a HTML Doxygen dokumentaci firmware digitálního záznamníku
        \item součástí projektu jsou nutné externí knihovny a moduly, které jsou potřeba k sestavení projektu, tyto celky byly převzdaty z MCUXpresso SDK
    \end{itemize}
    \item \texttt{hardware/} -- adresář s projektem expanzní desky: schématy zapojení, návrhem PCB, seznamem součástek (BOM) a schématem v PDF

    \begin{itemize}
        \item \texttt{design/} -- adresář obsahující KiCAD projekt expanzní desky.
        \item \texttt{BOM.txt} -- adresář obsahující hlavičkové soubory firmware digitálního záznamníku
    \end{itemize}
    
    \item \texttt{power\_consumption/} -- data z analýzy spotřeby energie, změřená pomocí Power Profiler Kit II od společnosti Nordic Semiconductors

    \item \texttt{powerloss\_detection/} -- data z analýzy chování digitálního záznamníku při ztrátě napájecího napětí zaznamenaná pomocí Saleae Logic 16 Pro
    
    \item \texttt{tests/} -- adresář obsahující testovací skripty, testovací soubory a výstupy ze statické analýzy kódu pomocí nástroje PC-lint Plus

    \item \texttt{misc/} -- složka obsahující obrázky k README

    \item \texttt{thesis/} -- adresář obsahující text technické zprávy k bakalářské práci, včetně obrázků a diagramů
    
    \item \texttt{manual.pdf} -- manuál k použití digitálního záznamníku
    \item \texttt{video-ukazka.mp4} -- video ukázka použití digitálního záznamníku ve formátu MP4
    \item \texttt{video-ukazka.mov} -- video ukázka použití digitálního záznamníku ve formátu MOV
    \item \texttt{README.md} -- úvodní soubor s popisem projektu, instalací a způsobem použití
\end{itemize}

Celý obsah přiloženého paměťového média je rovněž dostupný prostřednictvím cloudového úložiště NextCloud, které poskytuje Fakulta informačních technologií VUT v Brně, na adrese: \url{https://nextcloud.fit.vutbr.cz/s/xgZG6ZdZeccKQMC}.

\chapter{Projekt digitálního záznamníku}
Digitální záznamník je možné dále rozšířit nebo upravit dle konkrétních požadavků. Celý projekt, včetně implementace firmwaru, návrhu expanzní desky i veškeré dokumentace, je veřejně dostupný na platformě GitHub.

\begin{itemize}
    \item Odkaz na GitHub repozitář je \url{https://github.com/Doly02/nxp-mcxn947-datalogger}
\end{itemize}

\chapter{Použité nástroje}

\begin{itemize}
    \item \textbf{MCUXpresso IDE} ve verzi 11.10.0, jež bylo využito jako hlavní vývojové prostředí pro implementaci firmwaru digitálního záznamníku.
    \item \textbf{NXP SDK FRDM-MCXN947} ve verzi 2.16.000 s~\textbf{GCC-based ARM Embedded Toolchain} včetně kterých jsou ovladače pro desku \textbf{FRDM-MCXN947}, knihovna \textbf{FATFS}, \textbf{FreeRTOS} a~moduly \textbf{Mass Storage}, \textbf{MMC} a~\textbf{USB}.
    \item \textbf{MCU-Link}, který byl použitý pro nahrávání a ladění firmwaru.
    \item \textbf{nRF Connect for Desktop} od \textbf{Nordic Semiconductor} pro měření spotřeby digitálního záznamníku pomocí \textbf{Power Profiler Kit II}.
    \item \textbf{Logic} ve verzi 2.4.22 pro pozorování signálů digitálního záznamníku pomocí logického analyzátoru \textbf{Saleae Logic Pro~16}.
    \item \textbf{GitHub} pro verzování firmwaru digitálního záznamníku a~technické zprávy.
    \item \textbf{KiCAD 8} pro návrh expanzní desky digitálního záznamníku.
    \item \textbf{PC-lint Plus} ve verzi 2.2 pro statickou analýzu zdrojového kódu. 
    \item \textbf{Doxygen} pro generování HTML a~LaTeX dokumentace zdrojového kódu.
    \item \textbf{Overleaf} pro úpravu zdrojových kódů technické zprávy.
    \item \textbf{Draw.io} pro tvorbu diagramů, které jsou součástí technické zprávy.
    \item \textbf{ChatGPT}, který byl využit pro návrhy vylepšení struktury a~formulace vět v~technické zprávě.
\end{itemize}

% Pro kompilaci po částech (viz projekt.tex) nutno odkomentovat
%\end{document}
