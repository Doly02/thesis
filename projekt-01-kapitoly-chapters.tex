% Tento soubor nahraďte vlastním souborem s obsahem práce.
%=========================================================================
% Autoři: Michal Bidlo, Bohuslav Křena, Jaroslav Dytrych, Petr Veigend a Adam Herout 2019

% Pro kompilaci po částech (viz projekt.tex), nutno odkomentovat a upravit
%\documentclass[../projekt.tex]{subfiles}
%\begin{document}

% TODO: Kde rict ze implementovany zaznamnik je konkretne na seriovou komunikaci - vylozene UART? 

\chapter{Úvod}
\label{uvod}
Tato bakalářská práce je věnována návrhu a implementaci digitálního záznamníku s autonomním dokončením záznamu dat při výpadku napájecího napětí. Požadavek na zařízení vznikl od firmy NXP 
Semiconductors, konkrétně od týmu zaměřeného věnující se bezdrátovému nabíjení, ve kterém pracuji. Tento tým působí v České republice, jak v Rožnově pod Radhoštěm tak i v Brně a zároveň 
má své zastoupení v Asii a Severní Americe. NXP Semiconductors je jedním z předních členů WPC (Wireless Power Consortium), organizace zodpovědné za definování standardu Qi pro bezdrátové 
nabíjení. Primární zaměření NXP v této oblasti spočívá ve vývoji referenčních designů pro automotive sektor, kde zákazníkům poskytuje řešení určená pro integraci do vozidel.

Zákazníci, kteří využívají referenční designy NXP, pocházejí z celého světa a dostávají téměř hotový produkt, který lze následně certifikovat v Qi certifikačních laboratořích. Nicméně i 
přesto, že jsou referenční designy navrženy podle nejnovějších standardů, často dochází k jejich úpravám podle specifických požadavků zákazníků, zejména s ohledem na konkrétní poptávku 
koncového zákazníka (OEM – Original Equipment Manufacturer). Tyto požadavky jsou obvykle shrnuty v RFP (Request for Proposal), kde zákazník specifikuje konkrétní požadavky na systém. 
Tyto úpravy mohou být například realizovány z důvodu snížení ceny nebo zlepšení výkonu, například EMC charakteristik a nebo speciální chování bezdrátové nabíječky v krajních situacích. 

Při jakýchkoli úpravách však vznikají nové technické výzvy, a proto NXP poskytuje zákazníkům plnou technickou podporu až do úspěšné certifikace. Certifikace probíhá v různých laboratořích 
po celém světě, avšak ne vždy může být přítomen zaměstnanec NXP, který by dohlížel na celý proces a zajistil, že certifikace proběhne hladce. V těchto případech se momentálně tým pro 
bezdrátové napájení spoléhá pouze na záznamy poskytnuté operátorem certifikační laboratoře. Tyto záznamy však pocházejí pouze ze strany přijímače – tedy certifikačního zařízení, zpravidla 
od výrobců Nok9 nebo Granite River Labs (GRL). Ty poskytují některé z důležitých informací, bohužel tyto nabídnuté záznamy nezahrnují explicitní informace o chování vysílače. Pokud tedy 
bezdrátová nabíječka, tedy bezdrátový vysílač nějakým testem neprojde, což se občas stává, je často náročné zpětně identifikovat příčinu problému. \cite{nxp_wireless_charging_team}

Nezbytným požadavkem na implementaci tohoto záznamníku je i jeho snadná obsluha, neboť zařízení bude poskytováno zákazníkům pro již zmíněné účely certifikace. V klasickém scénáři zákazník 
předá nabíječku i se záznamníkem operátorovi certifikační laboratoře, ten si ji připojí k testovanému zařízení. Po skončení testovacího dne operátor záznamník vrátí zákazníkovi, který jej 
následně připojí k počítači a odešle společnosti NXP Semiconductors získané záznamy.


\chapter{Záznam dat}
\label{zaznam_dat}

\section{Počátky záznamu dat}
\label{pocatky}
Lidstvo již od svých počátků mělo potřebu zaznamenávat data, neboť člověk mnohdy dokáže datům přiřadit sémantiku - jejich význam, a proměnit je tak v informace. Právě díky nim se lidé 
mohou učit z minulých zkušeností, předávat znalosti dalším generacím, organizovat a podpořit tak neustálý lidský pokrok. 

% Nicméně největším pokrokem ve zaznamenávání dat nastal s příchodem výpočetní techniky.
Po staletí byl záznam dat výhradně manuální. Informace se uchovávaly v rukopisech, knihách či na papírových svitcích, ať už formou psaného textu, nebo ručně zapisovaným výsledků pozorování. 
Přesnost a dostupnost těchto dat byla však omezena kapacitou lidské obsluhy a možnostmi mechanických nástrojů.

\begin{figure}[h] % obrazek patent
    \centering
    \includegraphics[width=0.40\textwidth]{obrazky-figures/first_chart_recorder.png}
    \caption{První skutečný grafický záznamník (Chart Recorder) patentovaný Williamem Henrym Bristolem v roce 1888 \cite{bristol_chart_recorders}}
    \label{fig:chart_recorder}
\end{figure}

% pripadne to upravit na: Stěžejní změna přišla ve 20. století s rozvojem automatických záznamníků
Stěžejní změna přišla ve 20. století s rozvojem elektrotechniky a nástupem automatických záznamníků, které umožnily jednoduchý sběr dat bez nutnosti lidského zásahu. Namísto ručně 
zapisovaných měření začaly být hodnoty přenášeny přímo do záznamových zařízení, která je dokázala systematicky uchovávat. \cite{origin_of_chart_recorders}

\section{Záznam dat v počátcích elektrotechniky}
\label{zaznam}
Prvními specializovanými záznamníky byly mechanické či elektromechanické zařízení, využívající principu analogového záznamu dat. Jejich primárním účelem bylo zaznamenávání fyzikálních 
veličin, jako je například teplota, tlak, vlhkost nebo vibrace. Tyto přístroje využívaly myšlenky mechanického pohyblivého pera, které převádělo naměřenou hodnotu fyzikální veličiny 
na samočinný pohyb. Pro realizaci tohoto pohybu bylo nutné nejprve převést měřenou fyzikální veličinu na mechanický posun. Například pro měření teploty se běžně využíval bimetalový pásek, 
složený ze dvou kovových materiálů s různou hodnotou teplotní roztažitelnosti. Při změně teploty docházelo k prohnutí pásku v důsledku rozpínání kovu, čímž bylo rozpohybováno mechanické pero, 
které zapsalo hodnotu na paměťové médium. 

% mechanicky -< samocinny
% pomocí kterého se zapisovala změřená hodnota na paměťové médiu (papírová páska nebo papírový buben)

\begin{figure}[h] % obrazek polygraf
    \centering
    \includegraphics[width=0.50\textwidth]{obrazky-figures/polygraaf.png}
    \caption{Ukázka zařízení patřící do skupiny analogových záznamníků - polygraf \cite{polygraph_picture}}
    \label{fig:polygraaf}
\end{figure}


Tyto přístroje jsou běžně používány od druhé poloviny 19. století. Pro již zmíněný záznam teploty lze například využít přístroj zvaný cirkulární grafový záznamník (Circular Chart Recorder), 
dále je hojně využíván polygraf, využívaný jako detektor lži. Značnou nevýhodou těchto záznamníků bývá typ paměťového média, na které probíhá zápis hodnot, nejčastěji jim je papírová páska 
nebo papírový buben. Tyto pásky musí být velice často měněny za nové, ještě nepopsané, jelikož výsledné záznamy by se, jinak staly značně nepřehledné, pokud by byly popsány 
vícekrát. \cite{origin_of_chart_recorders}

Největší nevýhodou analogových záznamových systémů je jejich vysoká specializace \footnote{Řešení jsou současně mnohdy optimalizovaná pro záznam konkrétního systému.} pro jediný konkrétní 
typ záznamu. Tyto záznamníky tedy nejsou snadno upravitelné pro jiné účely, na rozdíl od digitálních řešení, která umožňují flexibilnější přizpůsobení (například pouhou úpravou programu) 
k sledování monitorované soustavy. Často je v těchto případech nutné využít jiné analogové řešení. \cite{analog_signal_and_digital_signal_processing_Tel_System}

Další limitací těchto přístrojů bylo ruční vyhodnocování dat, což bylo mnohdy časově zdlouhavé a také náchylné k chybám. K správné interpretaci dat byla často potřeba zkušená obsluha a v 
některých případech i pomocné měřící pomůcky. Přenos souborů a automatizace taktéž nebyla možná, proto jakmile se v polovině 20. století začaly na trh dostávat číslicové systémy, analogové 
záznamové systémy jimi byly postupně nahrazovány. \cite{newcastle_history_of_digital_computers, florian_prechod_a_analog_do_digital}

%Jak obecne probiha princip digitalniho zaznamu dat, co jsou to digitalni data, ze se v dnesni dobe uprednostnuje tento zpusob misto analogoveho zaznamu.

% TODO: Kde napsat co je to vubec zaznamnik? Ze je to pristroj, ktery zpracovava, uklada a pripadne analyzuje data...
\section{Digitální zpracování dat}
\label{digitalni_zaznam_dat}
S nástupem číslicových systémů v polovině 20. století došlo k velkému pokroku ve způsobu, jakým jsou data zaznamenávána, zpracovávána a uchovávána. Digitální záznam dat postupně nahradil 
analogové metody, které byly omezené nejen kapacitou paměťových médií (viz. kapitola \ref{zaznam}), ale také nutností manuálního vyhodnocení záznamů a obtížným sdílením získaných dat.

\subsection{Princip digitálního záznamu dat}
Digitální záznam dat se oproti analogovému liší způsobem, jakým se v systému obecně pracuje se signály (viz. \ref{zaznam}). Zatímco analogový záznam pracuje se spojitými (kontinuálními) 
signály, digitální záznam využívá diskrétní hodnoty, které jsou uchovávány v binární podobě. To znamená, že již na vstupu musí být příchozí signály v digitální podobě. Data tedy musí být 
generována číslicovými zdroji a nebo musí být převedena do digitálního tvaru pomocí komponenty k tomu určené - digitálně-analogového převodníku. 

V případě převodu analogových signálů do jejich digitální podoby prochází proces digitalizace ve třech základních krocích. V počátku dochází ke vzorkování, při kterém je tento spojitý signál 
snímán v pravidelných časových intervalech a převáděn na diskrétní hodnoty. Následně dochází ke kvantizaci, při níž jsou vzorkované hodnoty zaokrouhleny na nejbližší úroveň v omezeném rozsahu, 
to s sebou nese drobnou ztrátu přesnosti. Nakonec je kvantizovaný signál kódován do binární podoby, umožňující jeho další zpracování, ukládání a přenos výpočetním strojem.

\begin{figure}[h]
    \centering
    \includegraphics[width=0.95\textwidth]{obrazky-figures/digital_vs_analog.png}
    \caption{Průběh analogového a digitálního signálu}
    \label{fig:digital-vs-analog}
\end{figure}

U digitálních signálů je proces záznamu výrazně jednodušší, protože již nevyžaduje žádnou digitalizaci. Digitální data vstupující do záznamníku v podobě datové toku (data stream) skrz 
přijímací periferie. Tyto periferie jsou specializované nikoliv na konkrétní fyzikální veličinu (jak tomu bylo u analogových záznamnových zařízení viz. kapitola \ref{zaznam}), ale na 
specifický komunikační protokol. Tedy jedna periferie může přenášet jak údaje o teplotě, vlhkosti, tak i cokoliv jiného  pokud je dodrženo správné komunikační rozhraní, přičemž povaha 
přenášených dat závisí na senzorech a zařízeních připojených k této periferii. Přijímaná data jsou tedy příjímací periferii zpracovávány přímo ve své binární podobě, čímž odpadá celý 
proces potřeby provedení procesu digitalizace složené z vzorkování, kvantizace a kódování. 

\newpage

Digitální záznam poskytuje mnoho výhod oproti svému analogovému protějšku. Primárně je to jeho flexibilita a efektivita při zpracování, ukládání a přenosu dat. Digitální data lze snadno 
kopírovat, bezeztrátově přenášet a ukládat bez degradace kvality, což je pro záznamové systémy zásadní. Díky digitálnímu záznamu můžeme dnes i jednoduše analyzovat data, než tomu bylo dřív 
u analogového záznamu. Proto jsou dnes systémy s digitálním záznamem preferovanou volbou.

    
% TODO: Digitální záznamník může sloužit i pouze pro čistě PC programy? Pokud je zde nejak prijimaci periferie? To by slo primo do CPU
\subsection{Digitální záznamník}
\label{digitalni_zaznamik}
Digitální záznamník je dedikované zařízení nebo softwarový program určený ke sběru, zpracování a ukládání dat ve formě digitálního záznamu. Při pohledu na blokové diagramy digitálních 
záznamníků lze jejich strukturu rozdělit obecně do tří základních komponent, kterými jsou přijímající periferie, procesorové jádro a úložiště 
(viz. obrázek \ref{fig:common-digital-datalogger}). \cite{researchgate_general_dataloggger_multiple_sdcards, ieee_digital_sound_recorder_arm_sd_card, ieee_multi_connectivity_datalogger_sd_card}

Prvním klíčovým prvkem digitálního záznamníku je přijímající periferie, která slouží ke sběru vstupních dat. V závislosti na konkrétní aplikaci může tato komponenta zahrnovat různé 
typy vstupních rozhraní, jako jsou sériová rozhraní (UART, SPI, I2C a další) síťová rozhraní (Ethernet, Wi-Fi, LoRa, či CAN) nebo analogově-digitální převodníky.
\footnote{Vyjímkou jsou specifické monitorovací programy mezi, které patří například Windows Task Manager, jež ke svému sběru dat využívají rozhraní pro programování aplikací tzv. kernel 
API. \cite{fourcore_win_process_birth}} \cite{ieee_digital_sound_recorder_arm_sd_card}


Druhým hlavním prvkem digitálního záznamníku je procesorové jádro, zajišťující zpracování vstupních dat. Procesorové jádro může být jak součástí mikrokontroléru, tak i procesoru, které 
může mít v tomto případě na starosti jednoduché operace, jako přepočet hodnoty z analogově-digitálního převodníku na teplotu podle kalibrační křivky senzoru, přes filtrování šumu a 
doplňování časových značek k naměřeným vzorkům, až po pokročilé analýzy dat - například zpracování signálů pro EKG měření. \cite{springer_development_ECG_recorder}

Poslední a také jednou z nejdůležitějších všeobecnou částí digitálního záznamníku je úložiště, kde jsou data uložena pro pozdější přenos a zpracování (post-processing). Volba tohoto úložného 
prostoru závisí na požadavcích aplikace a rozsahu jejího využití - od osobních "hobby" projektů až po používání mnoha uživateli. V závislosti na tom lze využít různé technologie od paměťových 
karet SD a eMMC přes interní RAM či flash paměti až po síťová úložiště a cloudové služby. V mnoha případech je také využíván hybridní přístup, kdy jsou data nejprve ukládána do interní paměti 
záznamníku (například RAM úložiště) a následně dávkově přenášena na trvalé médium (viz. kapitola \ref{fig:batch-processing}) nebo odesílána přes síť. \cite{rta_local_vs_cloud}
    

\begin{figure}[h]
    \centering
    \includegraphics[width=0.95\textwidth]{obrazky-figures/common_digital_datalogger_scheme.png}
    \caption{Obecné schéma digitálního záznamníku}
    \label{fig:common-digital-datalogger}
\end{figure}

\newpage

Digitální záznamník poskytuje výstupy, kterými jsou organizovaná data, jež mohou být dále analyzována, vizualizována nebo zpracovávána jinými systémy. Jakou podobu mají výstupní data, tedy 
jejich formát, opět závisí na konkrétních požadavcích aplikace. Jedním je volba podle typu úložiště, jedná-li se o lokální úložný prostor, například paměťovou kartu, využívají se obecně 
soubory různých formátů, jako třeba textové nebo binární formy. Zatímco cloudová řešení využívají objektové formáty nebo lehké databázové systémy. Dále záleží, jakým způsobem bude proveden 
přenos dat, pokud bude využita síťová komunikace třeba pomocí MQTT či HTTP, je vhodné data uspořádávat do serializované podoby, zatímco při zvolení přenosu po sériové lince je naopak 
vhodnější a efektivnější využít opět některý z binárních formátů. Důležitou roli hrají i požadavky na následné zpracování (post-processing) a interpretaci dat v jiných systémech či 
aplikacích k tomu určeným. Například pro nazírání na data z pohledu časových řad může být vhodné využít formáty, které jsou kompatibilní s například databázovým systémem TimescaleDB. 
V dalších případech může být efektivní využít knihovny pro zpracování a analýzu dat, například Pandas v prostředí Pythonu, které umožňují rychlou manipulaci s velkým objemem strukturovaných 
dat, v takových případech je tedy zase lepší strukturovat data podle formátu CSV. \cite{medium_optimalization_iot_data_storage_timescaledb}

Způsobů, jak lze digitální záznamník sestavit, existuje mnoho, přičemž volba konkrétní architektury závisí na požadavcích dané aplikace. Velice často se však skládá z výše představených 
komponent. 

\subsection{Digitální záznam v počítačovém systému}
Digitální záznamníky lze implementovat na počítačových systémech jako jsou osobní počítače či servery, kde představují softwarová řešení, sloužící ke sběru, zpracování a potenciální 
archivaci dat. Tyto systémy obvykle také vycházejí ze struktury obecného číslicového záznamníku popsaného v kapitole \ref{digitalni_zaznamik}. 

Vstupní data přicházejí často z periferií, jako je například síťová karta či sériové porty, ale mohou také pocházet ze "pseudo" zařízení obsahujících stavy aplikací běžících na daném 
počítači či speciálních API (např. již zmiňovaných kernel API). Procesor, který tato data příjímá, tak je obvykle nejen zpracovává, ale často i určitým způsobem vyhodnocuje, jelikož 
disponuje dostatečným výpočetním výkonem pro pokročilé operace. Výsledkem těchto úkonů procesoru nad daty bývají informace o aktuálním stavu sledovaného systému, které lze využít k 
monitorování a dalším rozhodovacím procesům. \cite{linux_in_action_log_and_monitoring}

% Wireshark: RAM -> Export
% Terminal or Serial Dataloggers (COM, RS232,...) -> Export to File, Excel or Database
% 
Podle požadavků aplikace a jejího zaměření se liší i způsob, jakým jsou data uchovávána. Mnohdy si tyto záznamníky odkládají data pouze dočasně do operační paměti RAM, to umožňuje 
sledovat pouze aktuální stav nebo krátkodobé trendy. Pro sledování dlouhodobých trendů je pak možné tyto záznamy exportovat na dlouhodobá uložiště, do různých typů souborů - CSV, XLS 
či speciálních formátů relevantních dané aplikaci. Také se zde velmi často uplatňuje koncept exportování do databázových systémů a cloudových služeb.

\begin{figure}[h]
    \centering
    \includegraphics[width=0.90\textwidth]{obrazky-figures/computer_recorder.png}
    
    \caption{Schéma pokročilého digitálního záznamníku síťové komunikace - Wireshark \cite{researchgate_wireshark_architecture, winpcap_architecture}}
    \label{fig:computer-recorder}
\end{figure}

% Energie a udrzba?
Tyto záznamníky jsou především implementovány na strojích s relativně výkonnými hardwarovými prostředky, což je činí až překvalifikovanými pro mnohé stroje. Kromě provozních nákladů, 
jako je spotřeba elektrické energie, je také vysoká pořizovací cena, neboť každý záznamník vyžaduje plnohodnotný počítač s dostatečným výpočetním výkonem, úložným prostorem a případným 
připojením k síti. Proto se softwarové záznamníky na počítačových systémech nejčastěji využívají jako doplňkový nástroj pro monitorování procesů, jako jsou TCP/IP záznamníky, pokročilé 
sériové záznamníky a terminály, či záznamníky speciálních rozhraní, jako je například HID (Human Interface Device). Typicky se nasazují tam, kde počítač plní jinou primární funkci, 
například jako server zpracovávající velké objemy síťového provozu nebo vývojové prostředí pro embedded systémy. V těchto případech se využívají k diagnostice, analýze logů běžících 
aplikací nebo jako terminál pro připojená zařízení, čímž rozšiřují možnosti sledování a ladění bez nutnosti pořizování specializovaného hardwaru. 

Před implementací či použití takového záznamníku, je tedy dobré si promyslet, zda účel záznamníku, jestli budeme zaznamenávat stav pokročilého zařízení, na které se vleze i implementovaný 
záznamník. Pro dedikované záznamníky je z hlediska ceny a efektivity často výhodnější využít specializované vestavěné systémy, jejichž hardware je navržen tak, aby výkonově odpovídal 
konkrétním potřebám aplikace.

% Používají se však trošku jiným způsob
%  Je nutné si uvědomit, že jádro mikrořadiče neposkytuje stejnou výpočetní sílu jako jádro procesoru.

\subsection{Digitální záznam na platformě mikrořadiče}
Digitální záznamníky nemusí být nutně implementovány na výkonných počítačových systémech, ale mohou být také realizovány jako vestavěné systémy postavené na mikrořadičích (MCU). 
Narozdíl od svých protějšků na PC jsou však zaměřena na oblasti, kde je potřeba určitým způsobem sbírat a ukládat data s minimálními nároky na spotřebu energie a výpočetní výkon. 
Proto své uplatnění nacházejí v průmyslové automatizaci, IoT aplikacích, zdravotnických zařízeních a dalších oblastech.

I tyto záznamníky také obvykle obsahují základní komponenty obecného záznamníku představeného v kapitole \ref{digitalni_zaznamik}, které je možno libovolně rozšířit. Data vstupují 
do záznamníku prostřednictvím přijímacích periferií, která mohou pocházet z různých senzorů ať už analogových či digitálních (tedy teplotních čidel, akcelerometrů nebo proudových 
snímačů a dalších) nebo z jiných pozorovaných zařízení. Vstupní periferie záznamníků mohou tvořit klasické komunikační rozhraní, jakými jsou UART, SPI, I2C či I3C, ale také bezdrátová 
rozhraní v podobě Wi-Fi, Bluetooth nebo analogově-digitální převodník.\footnote{Vstupních komunikačních je vícero, zde jsou zmíněna jen ty nejzákladnější} Tyto vstupní periferie mohou 
být přímo integrovány v mikrořadiči - například již uvedený analogově-digitální převodník, nebo mohou být připojeny externě ve formě samostatných modulů, které komunikují s MCU 
prostřednictvím již standardních rozhraní.

\begin{figure}[h]
    \centering
    \includegraphics[width=0.95\textwidth]{obrazky-figures/recorder_mcu.png}
    
    \caption{Architektura digitálního záznamníku pro měření teploty}
    \label{fig:mcu-recorder}
\end{figure}

Vstupní data jsou následně zpracována jádrem mikrořadiče. To může provádět základní operace, jako je převod číslicové hodnoty z ADC na teplotu, filtrování signálu, doplňování časových 
značek, nebo provádět lehce obtížnější operace, jako je výpočet tepu v případě měření srdečních aktivit a další. 

Následně jsou data ukládána do uložiště. Obvykle se používají externí nevolatilní uložiště, která zaručují perzistenci dat i po vypnutí záznamníku. Standardně se dnes používají různé 
typy SD karet, které poskytují relativně jednoduché připojení    přes rozhraní SPI nebo SDIO. Široce využívané jsou také paměti FRAM (Ferroelectric Random Access Memory) které kombinují 
výhody rychlého zápisu (pohybujícího se kolem 1–10 MB/s) a nízké spotřeby energie. Možné je také upřednostnit přístup se vzdáleným uložištěm v podobě databáze či cloudového systému.


\section{Koncepty využívané ke zpracování dat digitálních záznamníků} 
Digitální záznamníky často bývají implementovány na platformě MCU, která dokáže poskytnout dobrý kompromis mezi cenou a výkonem, nicméně je nezbytné zohlednit specifická omezení a 
vlastnosti daného mikrokontroléru. Oproti počítačovým systémům mají MCU omezené výpočetní a paměťové zdroje, což vyžaduje důkladný návrh architektury systému. Proto je i nutné volit 
takové metody, které mohou minimalizovat latenci, spotřebu energie a nároky na paměť, a zároveň zajistí spolehlivý provoz v reálném čase.

\subsection{Vícenásobná vyrovnávací paměť (multiple-buffering)}
Jedním z častých konceptů využívaných v implementaci digitálních záznamníků je použití vícenásobné vyrovnávací paměti. Tento koncept je převážně známý díky algoritmům využívaným v 
oboru počítačové grafiky. Grafický čip musí zpracovat velké množství dat za krátký časový úsek, proto algoritmus zpracování dat využívá dvě vyrovnávací paměti - přední vyrovnávací paměť, 
takzvaný front-buffer, jež je využívána pro zobrazení aktuálního snímku a zadní vyrovnávací paměť, ve které čip připravuje nový obsah. Výsledný obsah tak může být plynule vykreslen bez 
artefaktů a trhání. Jakmile je nový snímek kompletní, vyrovnávací paměti se prohodí. \cite{double_buffering_model}

Obdobný mechanismus se využívá i v digitálních záznamnících, kde slouží k zajištění kontinuálního sběru dat bez výpadků. Zatímco jeden buffer přijímá nová data ze vstupní periferie 
(například analogově-digitálního převodníku či jiných vstupů), druhý buffer je současně zpracováván nebo ukládán na úložné médium. Tím se minimalizuje riziko ztráty dat způsobené časovou 
prodlevou při jejich zpracování nebo zápisu.

\begin{figure}[h]
    \centering
    \includegraphics[width=0.95\textwidth]{obrazky-figures/multiple_buffering-1-1.png}
    
    \caption{Schéma principu práce s vícenásobnou vyrovnávací paměťí - náhodný stav}
    \label{fig:multiple-buffering-1}
\end{figure}

Důležitou vlastností tohoto algoritmu je jeho nízká operační režijní náročnost. Plynulý chod zpracování dat je zajištěn bez nutnosti fyzického přenosu obsahu mezi vyrovnávacími pamětmi. 
Místo toho se využívají ukazatele (pointery), které směřují na počáteční adresy jednotlivých bufferů. Jakmile je sběrný buffer (Back Buffer) naplněn, ukazatele se prohodí~–~back buffer 
se stane zpracovávaným bufferem (Front Buffer) a původní front buffer se uvolní pro další sběr dat.

\begin{figure}[h]
    \centering
    \includegraphics[width=0.90\textwidth]{obrazky-figures/multiple_buffering-2-2.png}
    
    \caption{Schéma principu práce s vícenásobnou vyrovnávací paměťí - přeřazení ukazatelů}
    \label{fig:multiple-buffering-2}
\end{figure}

Tato metoda nachází významné uplatnění zejména v systémech pracujících v reálném čase, kde dochází k příjmu velkého objemu dat v krátkých časových intervalech a kde doba zpracování nesmí 
překročit dobu sběru dat. Využitím vícenásobné vyrovnávací paměti se minimalizuje latence zpracování a současně se snižuje riziko přetečení paměťového prostoru. \cite{buffering_chang}

Nicméně, žádný algoritmus není dokonalý a i tato metoda má své nevýhody, které je třeba zmínit. Vyrovnávací paměti jsou zpravidla implementovány softwarově, nikoliv hardwarově, což vede 
k zvýšeným nárokům na paměťové prostředky, obzvlášť v segmentu volatilní paměti (SRAM/DRAM) kde jsou buffery uložené. \cite{basics_of_digital_forensics}
% TODO: Je tedy vhodne si promyslet zda zdroje MCU budou dostacovat.

\subsection{Dávkové zpracování (batch processing)}
\label{davkove_zpracovani}
Další princip, jež je využívaný v implementacích digitálních záznamníků, souvisí s typem uložišť, na které jsou získaná data zaznamenávána. Některé typy nevolatilních pamětí, například 
NAND Flash paměť, jež umožňují uchování dat i po odpojení napájení. Tyto druhy paměti jsou organizovány do bloků (viz. obrázek \ref{fig:batch-processing}), přičemž bloky jsou následně 
rozděleny na menší jednotky zvané sektory. Velikost sektoru obvykle bývá 512 bajtů či 4096 bajtů, v závislosti na typu média a jeho architektuře. Tato bloková struktura umožňuje účinnou 
správu prostoru, které uložiště nabízí, ale současně vyžaduje specifický způsob zápisu/čtení dat, které je pouze umožněno na úrovni celých bloků. \cite{tech_target_nand_flash, 
non_volatile_memories}

Dávkové zpracovátí tohoto chování paměti využívá, data se tedy nejprve shromažďují ve volatilní paměti - například RAM a teprve po naplnění určitého objemu (celého bloku či jeho násobku) 
dojde k jejich zápisu na konečné paměťové médium.

\begin{figure}[h]
    \centering
    \includegraphics[width=0.70\textwidth]{obrazky-figures/batch_processing.png}
    
    \caption{Organizace bloku nevolatilní paměti \cite{ieee_relationships_among_region_segment_frame_and_cluster}}
    \label{fig:batch-processing}
\end{figure}

\newpage

\subsection{Cirkulární buffer}
Cirkulární buffer (circular buffer), někdy také označovaný jako kruhový nebo cylindrický buffer, je datová struktura, která funguje na principu FIFO fronty (First-In, First-Out) a považuje 
tuto paměť za kruhovou. Tento přístup je často využíván k řešení problému jednoho producenta a konzumenta (producent-consumer problem), kde jedno vlákno je konzument a druhé producent. 
Například ve vestavěných zařízeních, jednímž vláken je rutina obsluhy přerušení, která čte data ze senzoru a druhým vláknem je hlavní smyčka události.  \cite{embedjournal_ring_buffer}


\begin{figure}[h]
    \centering
    \includegraphics[width=0.88\textwidth]{obrazky-figures/circular_buffer_titles.png}
    
    \caption{Cirkulární vyrovnávací paměť}
    \label{fig:circular-buffer}
\end{figure}

Princip činnosti cirkulárního bufferu spočívá v použití dvou ukazatelů - zápisový ukazatel (head) a čtecí ukazatel (tail). Ukazatel head vždy směřuje na pozici, kam bude zapisován následující 
prvek, zatímco ukazatel tail ukazuje na pozici, ze které bude čtena následující hodnota. Pokud ukazatel head dosáhne konce pole, vrací se na jeho začátek, čímž je zajištěna kruhová povaha 
struktury. Při plném bufferu lze zvolit dvě strategie – přepsání nejstarších dat nebo odmítnutí nových vstupů, přičemž výběr závisí na konkrétní aplikaci. 
\cite{embedjournal_ring_buffer, medium_ring_buffer}

Z hlediska časové složitosti nabízí cirkulární buffer konstantní časovou složitost - O(1) pro základní operace, jako je zápis (enqueue) a  čtení (dequeue). Tato efektivita je dána tím, 
že se při zápisu a čtení dat není potřeba přesouvat prvky v paměti, ale lze pouze inkrementovat ukazatele s využitím operace modulo. Pokud jde o prostorovou složitost, velikost 
cirkulárního bufferu je určena předem – zpravidla jde totiž o staticky alokované, to v tomto případě odpovídá složitosti O(n), kde n je maximální počet prvků, které může buffer pojmout. 
\cite{petrungaro_ring_buffer_complexity}

\subsection{Nízko-energetické režimy (low-power modes)}
\label{nizko_energeticke_rezimy}
Energetická efektivita je jedním z klíčových parametrů obecně vestavěných zařízení, tedy i digitálních záznamníků implementovaných na platformě MCU, zejména pokud jsou napájeny z baterií 
či jiných omezených zdrojů energie (například energy harvesting). Minimalizace spotřeby bývá v těchto případech realizována využitím nízkoenergetických režimů (low-power modes), které 
umožňují zařízení přejít do stavu s minimální energetickou náročností během nečinných period. V praxi mnoho digitálních záznamníků nemusí provádět měření a záznam dat nepřetržitě. Například 
záznamník teploty může v pravidelných intervalech provést měření, uložit naměřenou hodnotu, přejít do režimu nízké spotřeby a po uplynutí definovaného časového intervalu nebo při vyskytnutí 
speciální události přejít do aktivního režimu. \cite{analog_devices_low_power_modes}

Průběh takového cyklického chování spotřeby mikrokontroléru, kde se střídají fáze měření a spánku s pravidelnou periodou měření teploty, je znázorněn na obrázku \ref{fig:low-power-modes} níže.

\begin{figure}[h]
    \centering
    \includegraphics[width=0.83\textwidth]{obrazky-figures/low_power_modes.png}
    
    \caption{Graf znázorňujíci dynamiku spotřeby mikrokontroléru v průběhu času při využití aktivního a nízkoenergetického režimu}
    \label{fig:low-power-modes}
\end{figure}

Ačkoliv nízkoenergetické režimy přinášejí značné úspory energie a jsou nezbytné pro zařízení napájená z baterií, u dataloggerů s velkým objemem zaznamenávaných dat mohou představovat významná 
omezení. Tyto režimy sice snižují energetickou náročnost systému, avšak zároveň omezují schopnost mikrokontroléru rychle reagovat na události. Spánkové stavy, které minimalizují spotřebu 
energie, často vedou k delšímu zpoždění při probuzení a nižší dostupnosti kritických periferií. V aplikacích, kde je vyžadována okamžitá odezva na externí podněty nebo nepřetržité zpracování 
velkého množství dat, může tento faktor negativně ovlivnit spolehlivost a efektivitu záznamníku. \cite{embedded_low_power_modes}

V těchto případech je proto nutné zvážit provozní podmínky a očekávanou dostupnost systému. Pokud záznamník pracuje s velkým datovým tokem a má možnost být připojen po dobu záznamu stále k 
externímu napájení, může být výhodnější upustit od implementace nízkoenergetických režimů a místo toho optimalizovat architekturu systému pro nepřetržitý provoz s důrazem na výkon a rychlou 
odezvu. \cite{analog_devices_low_power_modes}

% Normostrany = 11.34

\section{Způsoby zápisu dat}
Stejně jak jsou důležitá uložiště, na které jsou zaznamenaná data získana, je tak je důležitá práce s tímto uložištěm. Záznamníky dat musí být navrženy tak, aby umožnily spolehlivé ukládání 
získaných dat, které by mělo být efektivní ve smyslu rychlosti a šetrné pro zvýšení životnosti uložiště. Tato kapitola popisuje tři různé způsoby ukládání dat: přímý zápis na lokální úložiště, 
ukládání prostřednictvím mezivrstvy s FRAM pamětí a využití vzdálených úložišť. Každá z těchto metod má své specifické výhody a omezení, které určují její vhodnost pro konkrétní aplikace.

\subsection{Přímý zápis na permanentní uložiště}
Přímý zápis na permanentní úložiště představuje nejjednodušší a nejpřímější metodu ukládání dat. V tomto případě jsou zaznamenaná data ihned zapisována na nevolatilní paměťové médium, jako je 
SD karta, eMMC, USB Flash disk nebo NAND Flash čip. Tento způsob eliminuje potřebu mezivrstvy mezi záznamníkem a úložištěm, čímž se minimalizuje latence a zjednodušuje celková implementace.

\begin{figure}[h]
    \centering
    \includegraphics[width=0.85\textwidth]{obrazky-figures/forward_write.png}
    
    \caption{Přímý zápis na permanentní uložiště s SDHC kartou za pomocí čtyř pinové datové sběrnice}
    \label{fig:forward-write}
\end{figure}

Hlavní výhodou této metody je její jednoduchost a okamžitá perzistence dat. Data jsou ukládána přímo na trvalé úložiště a nehrozí tak jejich ztráta při výpadku napájení. To je zvláště výhodné 
v záznamnících, které zapisují drobné množství dat, kde se data neměří neustále, ale jednou za danou periodu uvedenou v kapitole \ref{nizko_energeticke_rezimy}, například v environmentálních 
záznamnících monitorujících teplotu nebo vlhkost.

Problémem tohoto principu je však častý zápis na paměťové médium. Proto, aby se předešlo nadměrnému opotřebení uložiště a zvýšila se efektivita zápisu, využívá se mnohdy současně metoda 
dávkového zpracování (batch processing) zmíněná v kapitole \ref{davkove_zpracovani}. Data jsou krátkodobě uložena ve volatilním uložišti, a jakmile jich je nashromažděno dostatek, tak jsou 
přepsána do dlouhodobé nevolatilní paměti. To ale přináší i nové úskalí, hodnoty uložené v neperzistentním uložišti jsou vystavena riziku ztráty v případě ztráty napájecího napětí.

\subsection{Zápis na permanentní uložiště přes mezivrstvu s FRAM paměti}
Alternativní volbou k přímému zápisu na permanentní úložiště je využití mezivrstvy ve formě FRAM (Ferroelectric Random Access Memory). FRAM je nevolatilní paměť, jež kombinuje výhody rychlé 
volatilní RAM paměti a perzistentního úložiště.

Ve feroelektrické RAM paměti (FRAM/FeRAM) jsou data ukládána pomocí změny polarizace feroelektrického materiálu v paměťové buňce. Jednotlivé buňky se skládají podobně jako je tomu u dynamické 
RAM (DRAM), z jednoho tranzistoru a jednoho kondenzátoru (1T-1C). Na rozdíl však od DRAM, kde je informace uchovávána jako elektrický náboj v lineárním dielektriku, FeRAM využívá feroelektrický 
materiál, jakým je třeba titaničitan olovnatý (PZT), který vykazuje hysterezní chování. Jakmile je aktivní elektrické pole, dipóly se v krystalové mřížce přeuspořádají do jednoho ze dvou 
stabilních stavů odpovídajících binárním hodnotám nula či jedna a tento stav zůstává zachován i po odeznění elektrického pole. \cite{ieee_feram_ultra_high_density_embedded_mem}

\begin{figure}[h]
    \centering
    \includegraphics[width=0.70\textwidth]{obrazky-figures/feram.png}
    
    \caption{Struktura feroelektrické RAM paměti (FeRAM) \cite{researchgate_nonvolatile_memory_technologies}}
    \label{fig:feram-structure}
\end{figure}

FRAM lze v dedikovaném digitálním záznamníku využít jako takzvanou mezivrstvu neboli vyrovnávací paměť, pomocí které lze optimalizovat zápisy na konečné dlouhodobé úložiště. Jak jsou tedy 
data záznamníkem postupně sbírána, tak mohou být postupně či po blocích zapisována do této mezivrstvy. Pokud je následně vyrovnávací paměť FRAM dostatečně zaplněna, její obsah je dávkově 
přenesen na dohodnuté trvalé úložiště, a tento cyklus se opakuje. 

% navíc nevolatilní, diky usporadanym dipolum, ktere bylo zajisteno aktivnim elektrickem poli 
Použití FRAM jako vyrovnávací paměti přináší několik výhod. Jedním z hlavních přínosů je snížení opotřebení hlavního úložiště, eliminán je totiž častý zápis po malých blocích dat, který 
zbytečně opotřebovává koncová nevolatilní úložiště typu flash, která mají omezený počet přepisovacích cyklů. Další výhodou je velice rychlý zápis oproti flash pamětem, obvykle trvá zápis 
na FRAM v řádu nanosekund, což je řádově rychlejší než srovnávané flash paměti, na kterých zápis trvá typicky v řádu mikrosekund až milisekund. FRAM je navíc nevolatilní, což znamená, že 
i v případě výpadku napájecího napětí zůstávají zaznamenaná data na feroelektrickém uložišti zachována, čímž se eliminuje potřeba dodatečných opatření k ochraně dat, jako jsou záložní 
baterie nebo superkondenzátory.

\subsection{Zápis na vzdálené uložiště}
\label{zapis_na_vzdalene_uloziste}
Pro dlouhodobé uložení dat lze v neposledním případě také využít vzdálená uložiště, jakými jsou databáze či cloudy. Lze tak využít jednotného uložiště pro velké množství digitálních 
záznamníků a eliminovat tak potřebu lokálních, nevolatilních paměťových médií. Zmiňované jednotné uložiště může být jak centrální server, nebo distribuovaná síťová soustava, umožňující 
uložení daleko většího množství dat, než může nabídnout lokální uložiště.

% Neni to tak ze CoAP a MQTT jsou si rovny, MQTT se hodi spis na komunikaci, kde Tx a Rx jsou synchronizovany, kde komunikace probiha casto asynchronne, jedno zarizeni data vysila a ostatni je mohou odebirat a nasledne prijimat, takhle jde udelat efektivni komunikaci bez nutnosti synchronizace
% CoAP, je zase spis pro format dotaz-odpoved mimikuje HTTP
V praxi dedikovaný záznamník, který data buď přímo zpracovává, nebo je dočasně uchovává v volatilní paměti, využívá síťové rozhraní k jejich přenosu do vzdáleného úložiště. Přenos probíhá 
obvykle prostřednictvím aplikačních síťových protokolů postavených nad transportním protokolem TCP (Transmission Control Protocol), jakým je MQTT, nebo nad protokolem UDP, nad kterým je 
postaven třeba protokol CoAP. Každý z uvedených zástupců poskytuje trošku odlišnou funkcionalitu. MQTT je vhodnější pro komunikaci, kde odesílatel a příjemce jsou synchronizováni a dialog 
probíhá asynchronně. Vysílatel tedy odesílá data a ostatní zařízení mohou začít. Zato CoAP zase prosazuje formát komunikace typu dotaz-odpověď, kterým mimikuje HTTP. \cite{emq_mqtt_vs_coap}


\begin{figure}[h]
    \centering
    \includegraphics[width=1.00\textwidth]{obrazky-figures/advanced_architecture_of_datalogging_3.png}
    
    \caption{Schéma pokročilého digitálního záznamníkíku s cloudovým uložištěm postavený na streamingové platformě Kafka \cite{confluent_advanced_datalogging, influxdata_advanced_datalogging_mmqt}}
    \label{fig:advanced-architecture-of-datalogging}
\end{figure}

% Pouzivaji se treba v pokrocilych systemech kdy uz je mozne z dat neco i odvozovat, napriklad venku je pekne a v dobe je 25 stupnu tak trosku pootevru okna.

Následně jsou data ukládány na již zmiňované databázové servery či cloudové služby. Databázové servery mohou být postaveny na různých technologiích v závislosti na typu dat a požadavcích 
na jejich zpracování. Často jsou využívány systémy, které umožńují pracovat s daty ve formátu časových řad, což jsou sekvence datových bodů zaznamenávanách ať už v pravidelných či 
nepravidelných časových intervalech. Na takové data může být tedy vhodné využít například InfluxDB nebo TimescaleDB.\footnote{Možné je také zvolit relační databáze, ale ty jsou u digitální 
záznamníků méně časté.} Cloudová řešení jsou pak objektová uložiště, kde jsou data, ukládány v podobě souborů či binárních objektů. Zárověn tyto služby umožňují krom samotného uložení, 
přidat i analytickou vrstvu, která umožňuje zpracování dat v reálném čase a reagovat tak na aktuální stav. K tomuto účelu pak lze využít třeba streamovací platformy jakámi jsou  Apache 
Kafka (viz. obrázek \ref{fig:advanced-architecture-of-datalogging} a AWS Kinesis. \cite{springer_analysis_time_series_db_edge_computing}

Jaké jsou tedy výhody tohoto přístupu? Především je to možnost centralizovaného ukládání a zpracování dat, to se hodí při velkých množstvích digitálních záznamníků, jelikož v takovémto 
případě nechceme obcházet všechna zařízení a postupně z nich extrahovat získaná data. Další výhodou je možnost reagovat na aktuální stav či odvozovat některé skutečnosti. Je tedy třeba možné 
v chytrých domácnostech automaticky upravit výkon klimatizace nebo vytápění na základě dat ze senzorů teploty a vlhkosti. \cite{springer_analysis_time_series_db_edge_computing}

Tento přístup se hodí pro digitální záznamníky operující ve známých prostředích, jakým je třeba zmiňovaná chytrá domácnost či továrna, jelikož je třeba zaručit stabilní připojení k síti. 
Pokud by měl záznamník cestovat různě po světě, bylo by třeba ho na každém novém místě připojit k Wi-Fi síti či Ethernetu, nebo by bylo třeba koncipovat tento digitální záznamník se SIM 
kartou, pomocí které by byl zajištěn přístup k mobilní síti, nicméně i to má své neduhy. Dále je důležité mít i koncipovanou komplexní infrastrukturu, jelikož jsou přenášena data po síti, 
je třeba přidat další úroveň zabezpečení, která zajistí autentizaci, šifrování, integritu dat a případně další bezpečnostní prvky. Dále je kolikrát možné tento přístup využít na zařízení, 
která že zařízení musí mít kolikrát celistvý TCP/IP modul (stack), bez něhož by záznamník nemohl podporovat protokoly, jakým je třeba MQTT, jež je postavený nad transportním protokolem TCP 
a zároveň z důvodu zmiňovaného zabezpečení.

\chapter{Návrh digitálního záznamníku}


\section{Existujících řešení digitálních záznamníku}
Také již bylo zmíněno, že záznam dat lidé řeší od nepaměti. A problém záznamníků dat se také nevyskytuje poprvé a existuje mnoho již produkčních řešení, která se zaměřují na záznam různých 
typů dat.Tato kapitola se věnuje popisu některých dostupných řešení, konkrétně záznamníkům zaměřeným na záznam datových toků zmíněných v kapitole \ref{digitalni_zaznam_dat}, mezi které se 
bude řadit i výsledné zařízení, které jsem navrhl, implementoval a popsal v této bakalářské práci. 


\subsection{OpenLog Serial Data Logger}
Prvním z vybraných existujících záznamníků je modul OpenLog, který je kompaktní a snadno použitelné zařízení pro záznam sériové komunikace, který vyvinula společnost SparkFun. Tento záznamník 
běží na osmibitovém mikrokontroléru ATmega328P taktovaném na 16 MHz a je navržen tak, aby umožnil přímý záznam sériové komunikace na microSD kartu s podporou až do velikosti 32 GB a bez 
nutnosti složitější konfigurace. \cite{sparkfun_openlog_tutorial}

\begin{figure}[h]
    \centering
    \includegraphics[width=0.85\textwidth]{obrazky-figures/sparkfun-openlog.png}
    
    \caption{Digitální záznamník SparkFun Openlog \cite{cirkit_openlog}}
    \label{fig:sparkfun-openlog}
\end{figure}

Tento záznamník je vhodný na jednoduché projekty, hobby projekty, či prototypování, OpenLog lze totiž snadno použít v systémech postavených například na nepájivém poli. není tedy potřeba 
složitá integrace do PCB. Záznamník je vhodný k monitorování systému, který již předzpracoval data ze zdroje dat (viz. obrázek \ref{fig:sparkfun-openlog-use-case}). Obvykle máme v tomto 
případě hlavní mikrokontroler, jež přijímá monitorovaná data (např. z GPS modulu) a následně je jakýmsi způsobem zpracovává a formátuje je do koncové podoby a následně je posílá po sériové 
lince na bázi UART komunikace, Openlog modulu. Záznamník si data následně ukládá do vyrovnávací paměti ve volatilní RAM paměti a jakmile se tento buffer naplní, tedy nasbírá 512 bajtů, tak 
je zapíše do micro SD karty. \cite{cirkit_openlog}

\begin{figure}[h]
    \centering
    \includegraphics[width=0.85\textwidth]{obrazky-figures/sparkfun-openlog-use-case.png}
    
    \caption{Sledovací zařízení GPS se záznamem dat \cite{cirkit_openlog}}
    \label{fig:sparkfun-openlog-use-case}
\end{figure}

Nevýhodou tohoto záznamníku je, že hlavní mikrokontrolér, který zpracovává data a posílá data následně do záznamníků, musí být aplikačně předchystán k tomuto účelu, což zvyšuje režii 
(overhead) navíc. Toto ovládací MCU musí buď pomocí příkazů nastavit parametry záznamu, jako je baudrate, vytvoření souboru, přidání dat do souboru, a další funkcionality, nebo je třeba 
využít konfigurační soubor uložený na SD kartě. Bohužel po nastavení v konfiguračním souboru je nutné provést napájecí cyklus (power cycle) než se nastavení propíše. Datalogger podporuje 
SD karty s kapacitou od 64 MB do 32 GB se souborovými systémy, nicméně je nutné tuto SD kartu zformátovat předem na zařízeních s operačním systémem Windows. Dále tento záznamník nepodporuje 
přístup na SD kartu bez nutnosti vyjmutí fyzického uložiště.

\subsection{Keeylog AirDrive Serial Logger}
\label{keelog_airdrive_serial_datalogger}
Dalším takovým řešením je AirDrive Serial Logger, vynalezený společností Keelog. Toto zařízení je moderní digitální záznamník poskytující bezdrátový přístup k datům, který umožňuje záznam 
sériových dat pomocí komunikačního standardu RS-232 či RS-485. Na rozdíl od tradičních řešení, která ukládají data pouze na lokální úložiště, nabízí tento záznamník konektivitu k Wi-Fi síti, 
čímž umožňuje vzdálený přístup k uloženým datům bez nutnosti vyjmutí fyzického média a manipulace s tímto zařízením. Keelog poskytuje vícero variant AirDrive záznamníků, které se liší 
poskytnutými funkcionalitami. Hlavním rozdílem je odlišný přístup k získaným datům, základní verze pracuje jako Wi-Fi hotspot, zatímco verze Pro a Max umožňují připojení do existující Wi-Fi 
sítě a také odesílání e-mailových reportů, časové razítkování záznamů nebo dokonce živé streamování dat. V čem se naopak zmíněné verze neliší, je velikostí interní paměti, která činí 16 GB, 
ta je přístupná i jako USB flash disk s rychlostí až 480 Mbps. Zařízení jako AirDrive nachází uplatnění zejména v průmyslovém monitorování, zpětném inženýrství sériových protokolů, zálohování 
dat z platebních terminálů nebo sběru dat ze senzorových systémů. \cite{keelog_airdrive_serial_datalogger, keelog_airdrive_serial_datalogger_max, keelog_airdrive_serial_datalogger_pro}

\begin{figure}[h]
    \centering
    \includegraphics[width=0.80\textwidth]{obrazky-figures/keeylog_airdrive_serial_logger-2.png}
    
    \caption{Keeylog AirDrive Serial Logger s přístupem k datům přes webové rozhraní \cite{keelog_airdrive_serial_datalogger, keelog_airdrive_serial_datalogger_scheme}}
    \label{fig:keelog-airdrive-serial-datalogger}
\end{figure}

Výhodou AirDrive Serial Loggeru je již zmiňovaná poskytovaná bezdrátová konektivita, která umožňuje přístup k datům z jakéhokoliv zařízení s Wi-Fi připojením, to se může hodit zejména v 
průmyslových  prostředích, kde může být velký počet takovýchto zařízení a získaná data tak mohou být hromaděna na jednom místě. Tímto centrálním bodem může být například cloudové úložiště 
nebo serverová databáze (viz. kapitola \ref{zapis_na_vzdalene_uloziste}), kam budou data pravidelně odesílána a následně mohou být analyzována. Tento záznamník také podporuje možnost 
konfigurace pomocí souboru CONFIG.TXT, ve kterém je možné nastavit, s jakou frekvencí budou získaná data odesílána do koncového úložiště. Pro a Max verze umožňují také nastavit tzv. živé 
vysílání (live streaming), při němž data mohou být monitorována a analyzována v reálném čase. Možné je také k datům přiřazovat časová razítka (timestamps), to se může hodit pro monitorování 
systémů, jejichž chování se chystáme porovnávat vůči jiným systémům a je tedy nutné si synchronizovat dva záznamy z různých zařízení. \cite{keelog_airdrive_serial_datalogger}

Navzdory svým pokročilým funkcím má AirDrive Serial Logger i několik nevýhod. Jednou z nich je omezení na standardy RS-232 a RS-485, které již nejsou v dnešní době široce rozšířené, a v 
mnoha systémech by bylo třeba využít sériové převodníky. Jelikož tyto standardy využívají asynchronní sériovou komunikaci (UART), není možné s tímto záznamníkem přímo zaznamenávat data z 
jiných běžných komunikačních sběrnic, jako jsou I2C, SPI, USB, CAN a ani nepodporuje možnost procesu digitalizace pomocí A-D převodníku. Tím se omezuje jeho univerzálnost a možnost použití 
v širším spektru aplikací. Další limitací je maximální přijímací přenosová rychlost UART (baud rate), která dosahuje pouze 115200 bps. To je například nevyhovující pro monitorování systémů 
bezdrátového nabíjení společnosti NXP Semiconductors, kde se komunikace probíhá s daleko vyšší komunikační rychlostí. \cite{keelog_airdrive_serial_datalogger}

\subsection{Anticyclone Systems AntiLog Data Logger Pro}
\label{anticyclone_systems_antilog_data_logger}
Třetím řešením je AntiLog Data Logger od společnosti Anticyclone Systems, který lze klasifikovat jako vysoce výkonný digitální záznamník určený pro záznam sériových dat v průmyslových a 
vývojových aplikacích.\footnote{Společnost Anticyclone Systems nabízí tři varianty těchto záznamníků, v tomto textu je primárně popsána verze Pro, jež je svými parametry a funkcionalitou 
nejblíže záznamníkům, které jsou předmětem této bakalářské práce.} Opět jako u řešení od společnosti Keyylog (viz. kapitola \ref{keelog_airdrive_serial_datalogger}) umožňuje data přijímat 
pomocí standardu RS-232 a také plnohodnotně zaznamenávat obousměrné sériové přenosy s vysokými přenosovými rychlostmi až 921 600 baudů. Zařízení umožňuje dlouhodobé zaznamenávání díky podpoře 
velkokapacitních nevolatilních úložišť až do velikosti 1 TB. Datalogger existuje v několika provedeních - verze AntiLog, AntiLog Pro a také OEM verze (ta je ve formě modulu), která umožňuje 
přímou integraci do jiných systémů. Nejpokročilejší verze Pro podporuje funkce jako časové razítkování (timestamps), podpora GNSS/NMEA dat a možnost vícekanálového záznamu, což jej činí 
atraktivní volbou pro aplikace, kde je potřeba přesné a rozsáhlé monitorování sériových přenosů. \cite{anticyclone_systems_antilog_pro}

\begin{figure}[h]
    \centering
    \includegraphics[width=0.6\textwidth]{obrazky-figures/antilogpro.png}
    
    \caption{Anticyclone Anti-Log Pro \cite{keelog_airdrive_serial_datalogger, keelog_airdrive_serial_datalogger_scheme}}
    \label{fig:sparkfun-openlog-use-case}
\end{figure}

Hlavní výhodou AntiLog Pro záznamníku je podpora vysokých přenosových rychlostí, což umožňuje záznam širokého spektra zařízení, která společně s nízkou spotřebou a možností přípojení 
baterie umožňují použití jak ve vnitřních prostředích, tak i v přírodě. Systém podporuje pokročilé časové razítkování (timestamping) s rezolucí až jednu milisekundu, ulehčující synchronizaci. 
Záznam lze rozšířit o měření veličin, jako je teplota, vlhkost či tlak, prostřednictvím podporovaných senzorů komunikujících po sběrnici I2C, a to paralelně se záznamem až dvou datových kanálů 
využívajících standard RS-232. Možné je také propojení až 255 jednotek do jednoho vícekanálového záznamníku, které pak umožňuje komplexní monitorování více zařízení 
současně. \cite{anticyclone_systems_antilog_pro, anticyclone_systems_antilog_pro_extended_logging}

Přesto má AntiLog Data Logger i své nevýhody. Velkým omezením je vysoká pořizovací cena, ta se aktuálně pohybuje u zařízení Anticyclone Anti-Log Pro od 229£ do 366£. Pro konfiguraci a 
přehrávání zaznamenaných dat je zase nutné využívat speciální aplikaci AntiTermPro RS-232 terminálový software, což znesnadňuje práci se zařízením pro nezkušenou 
obsluhu. \cite{anticyclone_systems_antilog_pro, anticyclone_systems_antilog_pro_price}

\section{Výběr vhodné platformy}
Již z předchozí kapitoly \ref{zaznam_dat} je jasné, že bude třeba implementovat digitální záznamník na mikrořadiči, jelikož požadavkem na koncový systém ze strany NXP Semiconductors je 
přenost, jednoduchost na použití nízká cena a něco vymysli. 

\subsection{FRDM-MCXN947}
Jednou z možností, jak digitální záznamník implementovat, je využití moderní platformy FRDM-MCXN947, což je vývojová deska společnosti NXP Semiconductors postavená na mikrořadiči MCXN947. 
Tato platforma je navržena pro vestavěné systémy s důrazem na výkon, rozšiřitelnost s rozumnou cenou.

Srdce mikrořadič MCXN947 je tvořeno dvoujádrovým procesorem Arm® Cortex®-M33, jež může být taktován až na 150 MHz.

tvoří 
Připravený i popis jádra ARM Cortex-M33, které využívá FRDM-MCXN947.

\begin{figure}[h]
    \centering
    \includegraphics[width=0.80\textwidth]{obrazky-figures/UM1208-blokovy-diagram-funkci.png}
    
    \caption{Keeylog AirDrive Serial Logger s přístupem k datům přes webové rozhraní \cite{keelog_airdrive_serial_datalogger, keelog_airdrive_serial_datalogger_scheme}}
    \label{fig:keelog-airdrive-serial-datalogger}
\end{figure}


\subsection{Arduino}

\subsection{Linux based - Raspberry}


% ----------------------------------------------------
% DALSI NAZVY: Volba datového úložiště, Výběr externího uložiště pro záznam dat, Možnosti způsobu ukládání získaných dat
\section{Přístupy k ovládaní úložiště}
Obecný popis, proč je potřeba externí uložiště, že by se data mohla ukládat i v RAM paměti, ale že by tam moc dlouho nevydržela, 

\subsection{SDIO}

\subsection{SPI}

\subsection{Quad-SPI flash}


% ----------------------------------------------------
% DALSI NAZVY: Volba datového úložiště, Výběr externího uložiště pro záznam dat
\section{Možnosti správy dat - souborové systémy} 
Obecný popis, souborový systém je zodpovědný za organizaci, správu a přístup k datům na zvoleném úložném médiu.

\subsection{FATFS}

\subsection{Chan FATFD}

\subsection{LittleFS}


% ----------------------------------------------------

\section{Výběr řízení přístupu k získaným datům}

\subsection{USB Mass Storage}

\subsection{Media Transfer Protocol}

\subsection{Human Interface Device}

% ----------------------------------------------------
\section{Výběr zdroje času}

\subsection{Obvod reálného času}

\subsection{Interní časovač}

\subsection{Bezdrátová komunikace (GPS/NTP)}

% ----------------------------------------------------
\section{Výber přístupu řízení běhu aplikace}
Srovnani obecne bare-metal a RTOS.


\subsection{Bare-Metal}

% TODO: Zde to asi spis rozdelit na Baremetall vs. RTOS a pak uvest priklady jako FreeRTOS a ZephyrRTOS
\subsection{RTOS}
\subsubsection{FreeRTOS}
Konkretní výhody a nevýhody FreeRTOS

\subsubsection{ZephyrRTOS}
Konkretní výhody a nevýhody ZephyrRTOS

% ----------------------------------------------------

\section{Architektura systému digitálního záznamníku}
Popis architektury na základě vybraných komponent. Popis blokového diagramu.

\begin{figure}[h]
    \centering
    \includegraphics[width=0.90\textwidth]{obrazky-figures/architecture-block-diagram.png}
    
    \caption{Výsledná architektura digitálního záznamníku}
    \label{fig:low-power-modes}
\end{figure}

% ----------------------------------------------------

\section{Volitelné rozšíření}
\subsection{Měření teploty}

% ----------------------------------------------------

\subsection{Řešení problému synchronizace času}

% ----------------------------------------------------

\chapter{Realizace hardwaru}
Popis co všechno je již na platformě FRDM MCXN947, z toho ti vyplyne co všechno bude ještě muset nabízet expanzní deska.
\section{Základová deska}
\section{Expanzní deska}

\section{Mechanická část}
Jak jsem měřil spotřebu, jak jsem spočítal hodnoty kondenzátorů

\chapter{Softwárová implementace}

\section{Záznamové vlákno}

\section{USB Mass Storage vlákno}

\section{Signalizace stavu systému}

% ==================================================== 
\chapter{Testování systému}
Obecne proc je potreba testovat, jake jsou moznost testovani a validace vestavenych systemu

% ----------------------------------------------------

\section{Testování a validace}

\subsection{Funkcionální testování}
Popis skriptů pro automatické testování, popis výsledků, atd, ...

\subsection{Kontrola bezpečnosti kódu}
MISRA

% ----------------------------------------------------

\section{Limitace systému}
\label{limitace}

% ----------------------------------------------------

\section{Možná rozšíření záznamníku}
\label{mozne_rozsireni}

\chapter{Závěr}
\label{zaverPrace}


%===============================================================================

% Pro kompilaci po částech (viz projekt.tex) nutno odkomentovat
%\end{document}
