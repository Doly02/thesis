% Tento soubor nahraďte vlastním souborem s obsahem práce.
%=========================================================================
% Autoři: Michal Bidlo, Bohuslav Křena, Jaroslav Dytrych, Petr Veigend a Adam Herout 2019

% Pro kompilaci po částech (viz projekt.tex), nutno odkomentovat a upravit
%\documentclass[../projekt.tex]{subfiles}
%\begin{document}

\chapter{Úvod}
\label{uvod}
Tato bakalářská práce je věnována návrhu a implementaci digitálního záznamníků s autonomním dokončením záznamu dat při výpadku napájení. Požadavek na zařízení vznikl od 
firmy NXP Semiconductors, konkrétně od týmu zaměřeného na bezdrátové nabíjení, ve kterém pracuji. Tento tým působí v České republice, jak v Rožnově pod Radhoštěm tak i v 
Brně a zároveň má své zastoupení v Asii a Severní Americe. NXP Semiconductors je jedním z předních členů WPC (Wireless Power Consortium), organizace zodpovědné za definování 
standardu Qi pro bezdrátové nabíjení. Primární zaměření NXP v této oblasti spočívá ve vývoji referenčních designů pro automotive sektor, kde zákazníkům poskytuje řešení 
určená pro integraci do vozidel.

Zákazníci, kteří využívají referenční designy NXP, pocházejí z celého světa a dostávají téměř hotový produkt, který lze následně certifikovat v Qi certifikačních 
laboratořích. Nicméně i přesto, že jsou referenční designy navrženy podle nejnovějších standardů, často dochází k jejich úpravám podle specifických požadavků zákazníků, 
zejména s ohledem na konkrétní poptávku koncového zákazníka (OEM – Original Equipment Manufacturer). Tyto požadavky jsou obvykle shrnuty v RFP (Request for Proposal), 
kde zákazník specifikuje konkrétní požadavky na systém. Tyto úpravy mohou být například realizovány z důvodu snížení ceny nebo zlepšení výkonu, například EMC charakteristik 
a nebo speciální chování bezdrátové nabíječky v krajních situacích. 

Při jakýchkoli úpravách však vznikají nové technické výzvy, a proto NXP poskytuje zákazníkům plnou technickou podporu až do úspěšné certifikace. Certifikace probíhá v 
různých laboratořích po celém světě, avšak ne vždy může být přítomen zaměstnanec NXP, který by dohlížel na celý proces a zajistil, že certifikace proběhne hladce. V těchto 
případech se momentálně tým pro bezdrátové napájení spoléhá pouze na záznamy poskytnuté operátorem certifikační laboratoře. Tyto záznamy však pocházejí pouze ze strany 
přijímače – tedy certifikačního zařízení, zpravidla od výrobců Nok9 nebo Granite River Labs (GRL). Ty poskytují některé z důležitých informací, bohužel tyto nabídnuté 
záznamy nezahrnují explicitní informace o chování vysílače. Pokud tedy bezdrátová nabíječka, tedy vysílač nějakým testem neprojde, což se občas stává, je často náročné 
zpětně identifikovat příčinu problému. \cite{nxp_wireless_charging_team}

Nezbytným požadavkem na implementaci tohoto záznamníku je i jeho snadná obsluha, neboť zařízení bude poskytováno zákazníkům pro účely certifikace. V klasickém scénáři 
zákazník předá nabíječku i se záznamníkem operátorovi certifikační laboratoře, ten si ji připojí k testovanému zařízení. Po skončení testovacího dne operátor záznamník 
vrátí zákazníkovi, který jej následně připojí k počítači a odešle společnosti NXP Semiconductors získané záznamy.


\chapter{Záznam dat}
\label{zaznam_dat}

\label{uvod}

\section{Počátky záznamu a zpracování dat (Historie záznamu a zpracování dat - předchozí název)}
\label{historie}
Lidstvo již od svých počátků potřebovalo zaznamenávat data, neboť člověk mnohdy dokáže datům přiřadit sémantiku - tedy význam, a proměnit je tak v informace. Právě díky nim 
se lidé mohou učit z minulých zkušeností, předávat znalosti dalším generacím, organizovat společenské a obchodní procesy a podporovat rozvoj vědy a technologií. Proto se 
již od pravěku hledaly způsoby, jak evidovat důležité události a hodnoty. První formy záznamu dat sahají až do doby 19 tisíc let před Kristem, kdy v paleolitu vznikl nástroj 
známý jako kost Ishango. Tento jednoduchý nástroj, vyrobený z kosti paviána, obsahoval vyryté zářezy, které pravděpodobně sloužily k provádění základních matematických 
operací, jako je sčítání či násobení.

\begin{figure}[h] % obrazek ishango
    \centering
    \includegraphics[width=0.6\textwidth]{obrazky-figures/ishango.jpg}
    \caption{Kost Ishango sloužící k záznamu informací v době Paleolitu \cite{ishango_picture}}
    \label{fig:ishango}
\end{figure}

S příchodem prvních civilizací se začaly vyvíjet lepší metody uchovávání dat. Ve starověké Mezopotámii, v oblasti mezi řekami Eufrat a Tigris, kde sídlil národ Sumérů, 
vzniklo kolem roku 3400 před naším letopočtem klínové písmo. Tento typ písma byl využíván hlavně pro zápis obchodních transakcí, zákonů a dalších důležitých informací, 
které byly zaznamenávány na hliněné tabulky pomocí rákosového stylusu. O několik století později, kolem roku 3200 př. n. l., začali Egypťané používat hieroglyfy, které 
byly ryty do kamene nebo zapisovány na papyrus, což umožnilo uchovávat důležité administrativní a náboženské záznamy. V Číně se mezi 1200–1000 př. n. l. používaly věštecké 
kosti, na které se zaznamenávaly otázky kladené orákulu - což jsou důležité otázky, zejména ve věcech budoucnosti, i jeho odpovědi, čímž vznikly první organizované záznamy 
psané ranými čínskými znaky.

\newpage

Průlomem pro zaznamenáváním dat, bylo vynalezení knihtisku, které vymyslel Johann Gutenberg ve 40. letech 15. století. Do této doby byly informace zaznamenávány ručním 
přepisováním, což bylo zdlouhavé a nákladné. V počátcích měl knihtisk význam převážně z náboženského hlediska, díky tisku se Bible mohla šířit mezi širokou veřejností, 
což přispělo k protestantské reformaci a změně křesťanského světa. Nicméně knihtisk obecně změnil způsob, jakým lidé přistupují k informacím a učinil je dostupnějšími. 
\cite{knihtisk_medium}

V 17. století se dále pokročilo, jak v způsobech záznamu dat, tak i s jejich zpracováním. Významným představitelem tohoto období je John Graunt, jež se věnoval studiu dat 
o úmrtnosti v Londýně, během čehož odhalil určité opakující se vzorce a zákonitosti na základě, kterých položil základy moderní statistiky. V následujících staletích 
pokračoval vývoj metod pro práci s daty, zejména díky rozvoji matematiky, fyziky a statistiky. \cite{britanicca_John_Graunt}
% Jako hlavni predstavitele tohoto obdobi lze uvest jmena jako Bayes, Laplace, Pierre-Simon, Carl Friedrich Gauss, Galton
\section{Záznam a zpracování dat v moderní době}
% TODO: pocet normostran (cca 3.1)
S pokročilými metodami na zpracování dat, také rostly nároky na výpočty a vznikla tak potřeba efektivnějších nástrojů pro zisk a analýzu dat. S tímto rozvojem souvisí i 
technologický pokrok v oblasti záznamu dat, který prošel dlouhou cestou – od jednoduchých mechanických přístrojů přes magnetické pásky až po moderní digitální záznamníky 
schopné uchovávat a analyzovat obrovské objemy informací v reálném čase.

%\section{Možnosti záznamu dat}
%\label{moznosti_zaznamu_dat}
  
\subsection{Analogový záznam dat} %TODO: zde popsat i analogový záznam i digitální záznamníky
\label{moznosti_zaznamu_dat}
Prvními specializovanými záznamníky byly mechanické či elektromechanické zařízení, využívající principu analogového záznamu dat. Jejich primárním účelem bylo zaznamenávání 
fyzikálních veličin, jako je například teplota, tlak, vlhkost nebo vibrace. Tyto přístroje využívaly myšlenky mechanického pohyblivého pera, které převádělo naměřenou 
hodnotu fyzikální veličiny na samočinný pohyb. Pro realizaci tohoto pohybu bylo nutné nejprve převést měřenou fyzikální veličinu na mechanický posun. Například pro měření 
teploty se běžně využíval bimetalový pásek, složený ze dvou kovových materiálů s různou hodnotou teplotní roztažitelnosti. Při změně teploty docházelo k prohnutí pásku v 
důsledku rozpínání kovu, čímž se rozpohybovalo mechanické pero, které zapsalo hodnotu na paměťové médium. 

% mechanicky -< samocinny
% pomocí kterého se zapisovala změřená hodnota na paměťové médiu (papírová páska nebo papírový buben)

\begin{figure}[h] % obrazek polygraf
    \centering
    \includegraphics[width=0.50\textwidth]{obrazky-figures/polygraaf.png}
    \caption{Ukázka zařízení patřící do skupiny analogových záznamníků - polygraf \cite{polygraph_picture}}
    \label{fig:polygraaf}
\end{figure}


Tyto přístroje byly běžně používány od 19. století. Pro již zmíněný záznam teploty se využíval například přístroj zvaný cirkulární grafový záznamník (Circular Chart Recorder), 
dále byl hojně využíván polygraf, využívaný jako detektor lži. Značnou nevýhodou těchto záznamníků byl typ paměťového média, na které probíhal zápis hodnot, nejčastěji jim 
byla papírová páska nebo papírový buben. Tyto pásky musely být velice často měněny za ještě nepopsané, jelikož výsledné záznamy by se staly značně nepřehledné, pokud by byly 
popsány vícekrát.
Další limitací těchto přístrojů bylo ruční vyhodnocování dat, což bylo mnohdy časově zdlouhavé a také náchylné k chybám. K správné interpretaci dat byla často potřeba 
zkušená obsluha a v některých případech i pomocné měřící pomůcky. Přenos souborů a automatizace také nebyla možná, proto jakmile se dostaly v polovině 20. století na trh 
číslicové systémy, začal jejich úpadek. \cite{newcastle_history_of_digital_computers, florian_prechod_a_analog_do_digital}

Jak obecne probiha princip digitalniho zaznamu dat, co jsou to digitalni data, ze se v dnesni dobe uprednostnuje tento zpusob misto analogoveho zaznamu.

% TODO: Kde napsat co je to vubec zaznamnik? Ze je to pristroj, ktery zpracovava, uklada a pripadne analyzuje data...
\section{Digitální záznam dat}
\label{digitalni_zaznamnik}
Co je to digitální záznamník + jak zaznamnik realizovat jestli PC nebo MCU (dedikovane zarizeni). Historie - odkdy se zacinaly pouzivat cislicove (digitalni) systemy, 
rozdily v zaznamu analogovych/digitalních dat

\begin{figure}[h] % obrazek polygraf
    \centering
    \includegraphics[width=0.95\textwidth]{obrazky-figures/common_digital_datalogger_scheme.png}
    \caption{Obecné schéma digitálního záznamníku}
    \label{fig:polygraaf}
\end{figure}
vstupy -> surové data (raw data)

vystupy -> organizovaná data např. v textové podobě

\subsection{Digitální záznam v počítačovém systému}
Ze je to pouze SW implementace zaznamniku, cilem je vyvinout pouze aplikaci pro zaznam dat, lze vybirat z daleko vice jazyku, K cemu jsou vyuzivany digitalni zaznamniky na PC.

\subsection{Digitální záznam na platformě mikrořadiče}

\section{Koncepty využívané ke zpracování dat digitálních záznamníků}
What is Lorem Ipsum?
Lorem Ipsum is simply dummy text of the printing and typesetting industry. Lorem Ipsum has been the industry's standard dummy text ever since the 1500s, when an unknown 
printer took a galley of type and scrambled it to make a type specimen book. It has survived not only five centuries, but also the leap into electronic typesetting, 
remaining essentially unchanged. It was popularised in the 1960s with the release of Letraset sheets containing Lorem Ipsum passages, and more recently with desktop 
publishing software like Aldus PageMaker including versions of Lorem Ipsum.

Why do we use it?
It is a long established fact that a reader will be distracted by the readable content of a page when looking at its layout. The point of using Lorem Ipsum is that it has 
a more-or-less normal distribution of letters, as opposed to using 'Content here, content here', 

\subsection{Vícenásobná vyrovnávací paměť (multiple-buffering)}
Jedním z častých konceptů využívaných v implementaci digitálních záznamníků je použití vícenásobné vyrovnávací paměti. Tento koncept je převážně známý díky algoritmům 
využívaným v oboru počítačové grafiky. Grafický čip musí zpracovat velké množství dat za krátký časový úsek, proto algoritmus zpracování dat využívá dvě vyrovnávací 
paměti - přední vyrovnávací paměť, takzvaný front-buffer, jež je využívána pro zobrazení aktuálního snímku 
a zadní vyrovnávací paměť, ve které čip připravuje nový obsah. Výsledný obsah tak může být plynule vykreslen bez artefaktů a trhání. Jakmile je nový snímek kompletní, 
vyrovnávací paměti se prohodí. \cite{double_buffering_model}

Obdobný mechanismus se využívá i v digitálních záznamnících, kde slouží k zajištění kontinuálního sběru dat bez výpadků. Zatímco jeden buffer přijímá nová data ze vstupní 
periferie (například analogově-digitálního převodníku či jiných vstupů), druhý buffer je současně zpracováván nebo ukládán na úložné médium. Tím se minimalizuje riziko 
ztráty dat způsobené časovou prodlevou při jejich zpracování nebo zápisu.

\begin{figure}[h]
    \centering
    \includegraphics[width=0.95\textwidth]{obrazky-figures/multiple_buffering-1.png}
    
    \caption{Schéma principu práce s vícenásobnou vyrovnávací paměťí - náhodný stav}
    \label{fig:multiple-buffering-1}
\end{figure}

Důležitou vlastností tohoto algoritmu je jeho nízká operační režijní náročnost. Plynulý chod zpracování dat je zajištěn bez nutnosti fyzického přenosu obsahu mezi 
vyrovnávacími pamětmi. Místo toho se využívají ukazatele (pointery), které směřují na počáteční adresy jednotlivých bufferů. Jakmile je sběrný buffer (Back Buffer) 
naplněn, ukazatele se prohodí~–~back buffer se stane zpracovávaným bufferem (Front Buffer) a původní front buffer se uvolní pro další sběr dat.

\begin{figure}[h]
    \centering
    \includegraphics[width=0.90\textwidth]{obrazky-figures/multiple_buffering-2.png}
    
    \caption{Schéma principu práce s vícenásobnou vyrovnávací paměťí - přeřazení ukazatelů}
    \label{fig:multiple-buffering-2}
\end{figure}

Tato metoda nachází významné uplatnění zejména v systémech pracujících v reálném čase, kde dochází k příjmu velkého objemu dat v krátkých časových intervalech a kde doba 
zpracování nesmí překročit dobu sběru dat. Využitím vícenásobné vyrovnávací paměti se minimalizuje latence zpracování a současně se snižuje riziko přetečení paměťového 
prostoru.

Nicméně, žadný algoritmus není dokonalý a i tato metoda mé své nevýhody, které je třeba zmínit. Vyrovnávací paměti jsou zpravidla implementovány softwárově, nikoliv 
hardwarově, což vede k zvýšeným nárokům na paměťové prostředky, obzvlášť v segmentu volatilní paměti (SRAM/DRAM) kde jsou buffery uložené. \cite{basics_of_digital_forensics}
% TODO: Je tedy vhodne si promyslet zda zdroje MCU budou dostacovat.

\subsection{Dávkové zpracování (batch processing)}
Další princip, jež je využívaný v implementacích digitálních záznamníků, souvisí s typem uložišť na které jsou získaná data zaznamenávána. Některé typy nevolatilních 
pamětí, například NAND Flash paměť, jež umožňují uchování dat i po odpojení napájení. Tyto druhy paměti jsou organizovány do bloků (viz. obrázek \ref{fig:batch-processing}), 
přičemž bloky jsou následně rozděleny na menší jednotky zvané sektory. Velikost sektoru obvykle bývá 512 bajtů či 4096 bajtů, v závislosti na typu média a jeho 
architektůře. Tato bloková struktura, umožňuje účinnou správu prostoru, které uložiště nabízí, ale současně vyžaduje specifický způsob zápisu/čtení dat, které je pouze 
umožněno na úrovni celých bloků.  \cite{tech_target_nand_flash}

Dávkové zpracovátí tohoto chování paměti využívá, data se tedy nejprve shromažďují ve volatilní paměti - například RAM a teprve po naplňení určitého objemu (celého bloku či 
jeho násobku) dojde k jejich zápisu na konečné paměťové médium.

\begin{figure}[h]
    \centering
    \includegraphics[width=0.80\textwidth]{obrazky-figures/batch_processing.png}
    
    \caption{Schéma organizace nevolatilní paměti}
    \label{fig:batch-processing}
\end{figure}



\subsection{Cirkulární buffer}

\subsection{Nízko-energetické režimy (low-power modes)}

\section{Typy médií pro záznam dat}
Hard disk, vyhody nevyhody,... 

\chapter{Návrh digitálního záznamníku}

\section{Existujících řešení digitálních záznamníku}

\section{Výběr vhodné platformy}
Neni vhodne implementovat zaznamnik na pocitacovem systemu, je treba zvolit stragii implementace na mikroradici, porovnat FRDM-MCXN947, Arduino, Raspberry
\subsection{FRDM-MCXN947}

\subsection{Arduino}

\subsection{Linux based - Raspberry}

Pripraven i popis jadra ARM Cortex-M33, ktere vyuziva FRDM-MCXN947.

% ----------------------------------------------------
% DALSI NAZVY: Volba datového úložiště, Výběr externího uložiště pro záznam dat, Možnosti způsobu ukládání získaných dat
\section{Přístupy k ovládaní úložiště}
Obecný popis, proč je potřeba externí uložiště, že by se data mohla ukládat i v RAM paměti, ale že by tam moc dlouho nevydržela, 

\subsection{SDHC}

\subsection{SPI}
\subsection{Quad-SPI flash}


% ----------------------------------------------------
% DALSI NAZVY: Volba datového úložiště, Výběr externího uložiště pro záznam dat
\section{Možnosti správy dat - souborové systémy} 
Obecný popis, souborový systém je zodpovědný za organizaci, správu a přístup k datům na zvoleném úložném médiu.

\subsection{FATFS}

\subsection{Chan FATFD}

\subsection{LittleFS}


% ----------------------------------------------------

\section{Výběr řízení přístupu k získaným datům}

\subsection{USB Mass Storage}

\subsection{Media Transfer Protocol}

\subsection{Human Interface Device}

% ----------------------------------------------------
\section{Výběr zdroje času}

\subsection{Obvod reálného času}

\subsection{Interní časovač}

\subsection{Bezdrátová komunikace (GPS/NTP)}

% ----------------------------------------------------
\section{Výber přístupu řízení běhu aplikace}
Srovnani obecne bare-metal a RTOS.


\subsection{Bare-Metal}

% TODO: Zde to asi spis rozdelit na Baremetall vs. RTOS a pak uvest priklady jako FreeRTOS a ZephyrRTOS
\subsection{RTOS}
\subsubsection{FreeRTOS}
Konkretní výhody a nevýhody FreeRTOS

\subsubsection{ZephyrRTOS}
Konkretní výhody a nevýhody ZephyrRTOS

% ----------------------------------------------------

\section{Architektura systému digitálního záznamníku}
Popis architektury na základě vybraných komponent. Popis blokového diagramu.

% ----------------------------------------------------

\section{Volitelné rozšíření}
\subsection{Měření teploty}

% ----------------------------------------------------

\subsection{Řešení problému synchronizace času}

% ----------------------------------------------------

\chapter{Realizace hardwaru}
Popis co všechno je již na platformě FRDM MCXN947, z toho ti vyplyne co všechno bude ještě muset nabízet expanzní deska.
\section{Základová deska}
\section{Expanzní deska}

\section{Mechanická část}
Jak jsem měřil spotřebu, jak jsem spočítal hodnoty kondenzátorů

\chapter{Softwárová implementace}

\section{Záznamové vlákno}

\section{USB Mass Storage vlákno}

\section{Signalizace stavu systému}

% ==================================================== 
\chapter{Testování systému}
Obecne proc je potreba testovat, jake jsou moznost testovani a validace vestavenych systemu

% ----------------------------------------------------

\section{Testování a validace}

\subsection{Funkcionální testování}
Popis skriptů pro automatické testování, popis výsledků, atd, ...

\subsection{Kontrola bezpečnosti kódu}
MISRA

% ----------------------------------------------------

\section{Limitace systému}
\label{limitace}

% ----------------------------------------------------

\section{Možná rozšíření záznamníku}
\label{mozne_rozsireni}

\chapter{Závěr}
\label{zaverPrace}


%===============================================================================

% Pro kompilaci po částech (viz projekt.tex) nutno odkomentovat
%\end{document}
