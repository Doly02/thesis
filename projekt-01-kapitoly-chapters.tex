% Tento soubor nahraďte vlastním souborem s obsahem práce.
%=========================================================================
% Autoři: Michal Bidlo, Bohuslav Křena, Jaroslav Dytrych, Petr Veigend a Adam Herout 2019

% Pro kompilaci po částech (viz projekt.tex), nutno odkomentovat a upravit
%\documentclass[../projekt.tex]{subfiles}
%\begin{document}

\chapter{Úvod}
\label{uvod}
Tato bakalářská práce je věnována návrhu a implementaci digitálního záznamníků s autonomním dokončením záznamu dat při výpadku napájení. Požadavek na zařízení vznikl od firmy NXP Semiconductors, 
konkrétně od týmu zaměřeného na bezdrátové nabíjení, ve kterém pracuji. Tento tým působí v České republice, jak v Rožnově pod Radhoštěm tak i v Brně a zároveň má své zastoupení v Asii a Severní 
Americe. NXP Semiconductors je jedním z předních členů WPC (Wireless Power Consortium), organizace zodpovědné za definování standardu Qi pro bezdrátové nabíjení. Primární zaměření NXP v této 
oblasti spočívá ve vývoji referenčních designů pro automotive sektor, kde zákazníkům poskytuje řešení určená pro integraci do vozidel.

Zákazníci, kteří využívají referenční designy NXP, pocházejí z celého světa a dostávají téměř hotový produkt, který lze následně certifikovat v Qi certifikačních laboratořích. Nicméně i přesto, 
že jsou referenční designy navrženy podle nejnovějších standardů, často dochází k jejich úpravám podle specifických požadavků zákazníků, zejména s ohledem na konkrétní poptávku koncového 
zákazníka (OEM – Original Equipment Manufacturer). Tyto požadavky jsou obvykle shrnuty v RFP (Request for Proposal), kde zákazník specifikuje konkrétní požadavky na systém. Tyto úpravy mohou 
být například realizovány z důvodu snížení ceny nebo zlepšení výkonu, například EMC charakteristik a nebo speciální chování bezdrátové nabíječky v krajních situacích. 

Při jakýchkoli úpravách však vznikají nové technické výzvy, a proto NXP poskytuje zákazníkům plnou technickou podporu až do úspěšné certifikace. Certifikace probíhá v různých laboratořích po 
celém světě, avšak ne vždy může být přítomen zaměstnanec NXP, který by dohlížel na celý proces a zajistil, že certifikace proběhne hladce. V těchto případech se momentálně tým pro bezdrátové 
napájení spoléhá pouze na záznamy poskytnuté operátorem certifikační laboratoře. Tyto záznamy však pocházejí pouze ze strany přijímače – tedy certifikačního zařízení, zpravidla od výrobců 
Nok9 nebo Granite River Labs (GRL). Ty poskytují některé z důležitých informací, bohužel tyto nabídnuté záznamy nezahrnují explicitní informace o chování vysílače. Pokud tedy bezdrátová 
nabíječka, tedy vysílač nějakým testem neprojde, což se občas stává, je často náročné zpětně identifikovat příčinu problému. \cite{nxp_wireless_charging_team}

Nezbytným požadavkem na implementaci tohoto záznamníku je i jeho snadná obsluha, neboť zařízení bude poskytováno zákazníkům pro účely certifikace. V klasickém scénáři zákazník předá nabíječku 
i se záznamníkem operátorovi certifikační laboratoře, ten si ji připojí k testovanému zařízení. Po skončení testovacího dne operátor záznamník vrátí zákazníkovi, který jej následně připojí 
k počítači a odešle společnosti NXP Semiconductors získané záznamy.


\chapter{Záznam dat}
\label{zaznam_dat}

\section{Počátky záznamu dat}
\label{pocatky}
Lidstvo již od svých počátků mělo potřebu zaznamenávat data, neboť člověk mnohdy dokáže datům přiřadit sémantiku - jejich význam, a proměnit je tak v informace. Právě díky nim se lidé mohou 
učit z minulých zkušeností, předávat znalosti dalším generacím, organizovat a podpořit tak neustálý lidský pokrok. 

% Nicméně největším pokrokem ve zaznamenávání dat nastal s příchodem výpočetní techniky.
Po staletí byl záznam dat výhradně manuální. Informace se uchovávaly v rukopisech, knihách či na papírových svitcích, ať už formou psaného textu, nebo ručně zapisovaným výsledků pozorování. 
Přesnost a dostupnost těchto dat byla však omezena kapacitou lidské obsluhy a možnostmi mechanických nástrojů.

\begin{figure}[h] % obrazek patent
    \centering
    \includegraphics[width=0.40\textwidth]{obrazky-figures/first_chart_recorder.png}
    \caption{První skutečný grafický záznamník patentovaný Williamem Henrym Bristolem \cite{bristol_chart_recorders}}
    \label{fig:chart_recorder}
\end{figure}

% pripadne to upravit na: Stěžejní změna přišla ve 20. století s rozvojem automatických záznamníků
Stěžejní změna přišla ve 20. století s rozvojem elektrotechniky a nástupem automatických záznamníků, které umožnily primitivní sběr dat bez nutnosti lidského zásahu. Namísto ručně zapisovaných 
měření začaly být hodnoty přenášeny přímo do záznamových zařízení, která je dokázala systematicky uchovávat. \cite{origin_of_chart_recorders}

\section{Záznam dat v počátcích elektrotechniky}
\label{zaznam}
Prvními specializovanými záznamníky byly mechanické či elektromechanické zařízení, využívající principu analogového záznamu dat. Jejich primárním účelem bylo zaznamenávání fyzikálních veličin, 
jako je například teplota, tlak, vlhkost nebo vibrace. Tyto přístroje využívaly myšlenky mechanického pohyblivého pera, které převádělo naměřenou hodnotu fyzikální veličiny na samočinný pohyb. 
Pro realizaci tohoto pohybu bylo nutné nejprve převést měřenou fyzikální veličinu na mechanický posun. Například pro měření teploty se běžně využíval bimetalový pásek, složený ze dvou kovových 
materiálů s různou hodnotou teplotní roztažitelnosti. Při změně teploty docházelo k prohnutí pásku v důsledku rozpínání kovu, čímž bylo rozpohybováno mechanické pero, které zapsalo hodnotu na 
paměťové médium. 

% mechanicky -< samocinny
% pomocí kterého se zapisovala změřená hodnota na paměťové médiu (papírová páska nebo papírový buben)

\begin{figure}[h] % obrazek polygraf
    \centering
    \includegraphics[width=0.50\textwidth]{obrazky-figures/polygraaf.png}
    \caption{Ukázka zařízení patřící do skupiny analogových záznamníků - polygraf \cite{polygraph_picture}}
    \label{fig:polygraaf}
\end{figure}


Tyto přístroje josu běžně používány od druhé poloviny 19. století. Pro již zmíněný záznam teploty se lze například využít přístroj zvaný cirkulární grafový záznamník (Circular Chart Recorder), 
dále je hojně využíván polygraf, využívaný jako detektor lži. Značnou nevýhodou těchto záznamníků byvá typ paměťového média, na které probíhá zápis hodnot, nejčastěji jim je papírová páska 
nebo papírový buben. Tyto pásky musí být velice často měněny za nové, ještě nepopsané, jelikož výsledné záznamy by se, jinak staly značně nepřehledné, pokud by byly popsány 
vícekrát. \cite{origin_of_chart_recorders}

Největší nevýhodou analogových záznamových systémů je jejich vysoká specializace \footnote{Řešení jsou současně mnohdy optimalizovaná pro záznam konkrétního systému.} pro jediný konkrétní typ 
záznamu. Tyto záznamníky tedy nejsou snadno upravitelné pro jiné účely, na rozdíl od digitálních řešení, která umožňují flexibilnější přizpůsobení (například pouhou úpravou programu) k 
sledování monitorované soustavy. Často je v těchto případech nutné využít jiné analogové řešení. \cite{analog_signal_and_digital_signal_processing_Tel_System}

Další limitací těchto přístrojů bylo ruční vyhodnocování dat, což bylo mnohdy časově zdlouhavé a také náchylné k chybám. K správné interpretaci dat byla často potřeba zkušená obsluha 
a v některých případech i pomocné měřící pomůcky. Přenos souborů a automatizace také nebyla možná, proto jakmile se dostaly v polovině 20. století na trh číslicové systémy, začal jejich 
úpadek. \cite{newcastle_history_of_digital_computers, florian_prechod_a_analog_do_digital}

%Jak obecne probiha princip digitalniho zaznamu dat, co jsou to digitalni data, ze se v dnesni dobe uprednostnuje tento zpusob misto analogoveho zaznamu.

% TODO: Kde napsat co je to vubec zaznamnik? Ze je to pristroj, ktery zpracovava, uklada a pripadne analyzuje data...
\section{Digitální záznam dat}
\label{digitalni_zaznam_dat}
S nástupem číslicových systémů v polovině 20. století došlo k velkému pokroku ve způsobu, jakým jsou data zaznamenávána, zpracovávána a uchovávána. Digitální záznam dat postupně nahradil 
analogové metody, které byly omezené nejen kapacitou paměťových médií (viz. kapitola \ref{zaznam}), ale také nutností manuálního vyhodnocení záznamů a obtížným sdílením získaných dat.

\subsection{Princip digitálního záznamu dat}
Digitální záznam dat se oproti analogovému liší způsobem zpracování signálů. Zatímco analogový záznam pracuje s kontinuálním (spojitým) signálem, jak bylo zmíněno v kapitole \ref{zaznam}, 
digitální záznam využívá diskrétní (číslicové) hodnoty. Tento rozdíl znamená, že analogový signál může nabývat libovolných hodnot v čase, zatímco digitální signál je reprezentován jako 
sada přesně definovaných úrovní, obvykle ve formě binárního kódu (nuly a jedničky).

\begin{figure}[h] % obrazek polygraf
    \centering
    \includegraphics[width=0.95\textwidth]{obrazky-figures/digital_vs_analog.png}
    \caption{Analogový a digitální signál}
    \label{fig:polygraaf}
\end{figure}

Hlavní výhodou digitálního záznamu oproti analogovému je jeho odolnost vůči šumu a zkreslení. Zatímco analogový signál může být postupně degradován například vlivem rušení nebo degradace 
záznamového média, digitální data lze bezztrátově kopírovat, přenášet a rekonstruovat bez snížení kvality.

\subsection{Digitální záznamník}
Digitální záznamník je zařízení nebo softwarový program určený ke sběru, zpracování a ukládání dat ve formě digitálního záznamu. Při pohledu na blokové diagramy digitálních záznamníků lze jejich strukturu rozdělit obecně do tří základních komponent, kterými jsou přijímající periferie, procesorové jádro a úložiště (viz. obrázek \ref{fig:common-digital-datalogger}). \cite{ieee_digital_sound_recorder_arm_sd_card, ieee_multi_connectivity_datalogger_sd_card, researchgate_general_dataloggger_multiple_sdcards}

Prvním klíčovým prvkem digitálního záznamníku je přijímající periferie, která slouží ke sběru vstupních dat. V závislosti na konkrétní aplikaci může tato komponenta zahrnovat různé typy vstupních rozhraní, jako jsou sériová rozhraní (UART, SPI, I2C a další) síťová rozhraní (Ethernet, Wi-Fi, LoRa, či CAN) nebo analogově-digitální převodníky.\footnote{Vyjímkou jsou specifické monitorovací programy mezi, které patří například Windows Task Manager, jež ke svému sběru dat využívají rozhraní pro programování aplikací tzv. kernel API. \cite{fourcore_win_process_birth}} \cite{ieee_digital_sound_recorder_arm_sd_card}

\newpage

Druhým hlavním prvkem digitálního záznamníku je procesorové jádro, zajišťující zpracování vstupních dat. Procesorové jádro může být jak součástí mikrokontroléru, tak i procesoru, které může mít v tomto případě na starosti jednoduché operace, jako přepočet hodnoty z analogově-digitálního převodníku na teplotu podle kalibrační křivky senzoru, přes filtrování šumu a doplňování časových značek k naměřeným vzorkům, až po pokročilé analýzy dat - například EKG měření. \cite{springer_development_ECG_recorder}

Poslední a také jednou z nejdůležitějších všeobecnou částí digitálního záznamníku je úložiště, kde jsou data uložena pro pozdější přenos a zpracování (post-processing). Volba tohoto úložného prostoru závisí na požadavcích aplikace a může zahrnovat různé technologie od paměťových karet SD a eMMC přes interní flash paměti až po síťová úložiště a cloudové služby.

\begin{figure}[h] % obrazek polygraf
    \centering
    \includegraphics[width=0.95\textwidth]{obrazky-figures/common_digital_datalogger_scheme.png}
    \caption{Obecné schéma digitálního záznamníku}
    \label{fig:common-digital-datalogger}
\end{figure}


\subsection{Digitální záznam v počítačovém systému}
Ze je to pouze SW implementace zaznamniku, cilem je vyvinout pouze aplikaci pro zaznam dat, lze vybirat z daleko vice jazyku, K cemu jsou vyuzivany digitalni zaznamniky na PC.

\subsection{Digitální záznam na platformě mikrořadiče}

\section{Koncepty využívané ke zpracování dat digitálních záznamníků}
Digitální záznamníky často bývají implementovány na platformě MCU, jak již bylo zmíněno. Realizace těchto zařízení vyžaduje pečlivý návrh, jelikož mikrořadiče poskytují více omezené zdroje 
než počítačový systém jak po paměťové, tak i po výpočetní stránce. 

\subsection{Vícenásobná vyrovnávací paměť (multiple-buffering)}
Jedním z častých konceptů využívaných v implementaci digitálních záznamníků je použití vícenásobné vyrovnávací paměti. Tento koncept je převážně známý díky algoritmům využívaným v 
oboru počítačové grafiky. Grafický čip musí zpracovat velké množství dat za krátký časový úsek, proto algoritmus zpracování dat využívá dvě vyrovnávací paměti - přední vyrovnávací paměť, 
takzvaný front-buffer, jež je využívána pro zobrazení aktuálního snímku a zadní vyrovnávací paměť, ve které čip připravuje nový obsah. Výsledný obsah tak může být plynule vykreslen 
bez artefaktů a trhání. Jakmile je nový snímek kompletní, vyrovnávací paměti se prohodí. \cite{double_buffering_model}

Obdobný mechanismus se využívá i v digitálních záznamnících, kde slouží k zajištění kontinuálního sběru dat bez výpadků. Zatímco jeden buffer přijímá nová data ze vstupní periferie (například 
analogově-digitálního převodníku či jiných vstupů), druhý buffer je současně zpracováván nebo ukládán na úložné médium. Tím se minimalizuje riziko ztráty dat způsobené časovou prodlevou při 
jejich zpracování nebo zápisu.

\begin{figure}[h]
    \centering
    \includegraphics[width=0.95\textwidth]{obrazky-figures/multiple_buffering-1.png}
    
    \caption{Schéma principu práce s vícenásobnou vyrovnávací paměťí - náhodný stav}
    \label{fig:multiple-buffering-1}
\end{figure}

Důležitou vlastností tohoto algoritmu je jeho nízká operační režijní náročnost. Plynulý chod zpracování dat je zajištěn bez nutnosti fyzického přenosu obsahu mezi vyrovnávacími pamětmi. 
Místo toho se využívají ukazatele (pointery), které směřují na počáteční adresy jednotlivých bufferů. Jakmile je sběrný buffer (Back Buffer) naplněn, ukazatele se prohodí~–~back buffer se 
stane zpracovávaným bufferem (Front Buffer) a původní front buffer se uvolní pro další sběr dat.

\begin{figure}[h]
    \centering
    \includegraphics[width=0.90\textwidth]{obrazky-figures/multiple_buffering-2.png}
    
    \caption{Schéma principu práce s vícenásobnou vyrovnávací paměťí - přeřazení ukazatelů}
    \label{fig:multiple-buffering-2}
\end{figure}

Tato metoda nachází významné uplatnění zejména v systémech pracujících v reálném čase, kde dochází k příjmu velkého objemu dat v krátkých časových intervalech a kde doba zpracování nesmí 
překročit dobu sběru dat. Využitím vícenásobné vyrovnávací paměti se minimalizuje latence zpracování a současně se snižuje riziko přetečení paměťového prostoru. \cite{buffering_chang}

Nicméně, žadný algoritmus není dokonalý a i tato metoda mé své nevýhody, které je třeba zmínit. Vyrovnávací paměti jsou zpravidla implementovány softwárově, nikoliv hardwarově, což vede 
k zvýšeným nárokům na paměťové prostředky, obzvlášť v segmentu volatilní paměti (SRAM/DRAM) kde jsou buffery uložené. \cite{basics_of_digital_forensics}
% TODO: Je tedy vhodne si promyslet zda zdroje MCU budou dostacovat.

\subsection{Dávkové zpracování (batch processing)}
Další princip, jež je využívaný v implementacích digitálních záznamníků, souvisí s typem uložišť na které jsou získaná data zaznamenávána. Některé typy nevolatilních pamětí, například NAND 
Flash paměť, jež umožňují uchování dat i po odpojení napájení. Tyto druhy paměti jsou organizovány do bloků (viz. obrázek \ref{fig:batch-processing}), přičemž bloky jsou následně rozděleny 
na menší jednotky zvané sektory. Velikost sektoru obvykle bývá 512 bajtů či 4096 bajtů, v závislosti na typu média a jeho architektůře. Tato bloková struktura, umožňuje účinnou správu prostoru, 
které uložiště nabízí, ale současně vyžaduje specifický způsob zápisu/čtení dat, které je pouze umožněno na úrovni celých bloků.  \cite{tech_target_nand_flash, non_volatile_memories}

Dávkové zpracovátí tohoto chování paměti využívá, data se tedy nejprve shromažďují ve volatilní paměti - například RAM a teprve po naplňení určitého objemu (celého bloku či jeho násobku) 
dojde k jejich zápisu na konečné paměťové médium.

\begin{figure}[h]
    \centering
    \includegraphics[width=0.70\textwidth]{obrazky-figures/batch_processing.png}
    
    \caption{Organizace nevolatilní paměti}
    \label{fig:batch-processing}
\end{figure}


\subsection{Cirkulární buffer}
Cirkulární buffer (circular buffer), někdy označovaný také jako kruhový nebo cylindrický buffer, je datová struktura, která funguje na principu kruhové fronty (FIFO – First-In, First-Out). 
Tento přístup je často využíván v systémech pro zpracování datových toků, jako jsou digitální záznamníky, kde je nezbytné kontinuálně přijímat data bez přerušení či ztráty vzorků.


\begin{figure}[h]
    \centering
    \includegraphics[width=0.60\textwidth]{obrazky-figures/circular_buffer.png}
    
    \caption{Cirkulární vyrovnávací paměť}
    \label{fig:circular-buffer}
\end{figure}

\newpage

Princip činnosti cirkulárního bufferu spočívá v použití dvou ukazatelů - head (zápisový ukazatel) a tail (čtecí ukazatel). Ukazatel head vždy směřuje na pozici, kam bude zapisován následující 
prvek, zatímco ukazatel tail ukazuje na pozici, ze které bude načtena následující hodnota. Pokud ukazatel head dosáhne konce pole, vrací se na jeho začátek, čímž je zajištěna kruhová povaha 
struktury. Při plném bufferu lze zvolit dvě strategie – přepsání nejstarších dat nebo odmítnutí nových vstupů, přičemž výběr závisí na konkrétní 
aplikaci.\cite{embedjournal_ring_buffer, medium_ring_buffer}

Z hlediska časové složitosti nabízí cirkulární buffer konstantní časovou složitost (O(1)) pro všechny základní operace, jako je zápis (enqueue) a  čtení (dequeue). Tato efektivita je 
dána tím, že se při zápisu a čtení dat není potřeba přesouvat prvky v paměti, ale stačí pouze inkrementovat ukazatele s využitím operace modulo. Pokud jde o prostorovou složitost, 
velikost cirkulárního bufferu je určena předem – zpravidla jde o staticky alokované pole o velikosti n prvků, což odpovídá složitosti O(n), kde n je maximální počet prvků, které může 
buffer pojmout. \cite{petrungaro_ring_buffer_complexity}

\subsection{Nízko-energetické režimy (low-power modes)}
Energetická efektivita je jedním z klíčových parametrů obecně vestavěných zařízení, tedy i digitálních záznamníků implementovaných na platformě MCU, zejména pokud jsou napájeny z 
baterií či jiných omezených zdrojů energie. Minimalizace spotřeby je v těchto případech realizována využitím nízkoenergetických režimů (low-power modes), které umožňují zařízení přejít 
do stavu s minimální energetickou náročností během nečinných period. V praxi mnoho digitálních záznamníků nemusí provádět měření a záznam dat nepřetržitě. Například záznamník teploty může 
v pravidelných intervalech provést měření, uložit naměřenou hodnotu, přejít do režimu nízké spotřeby a po uplynutí definovaného časového intervalu nebo při vyskytnutí speciální události 
přejít do aktivního režimu. \cite{analog_devices_low_power_modes}

Průběh takového cyklického chování spotřeby mikrokontroléru, kde se střídají fáze měření a spánku s pravidelnou periodou měření teploty, je znázorněn na obrázku \ref{fig:low-power-modes} níže.

\begin{figure}[h]
    \centering
    \includegraphics[width=0.90\textwidth]{obrazky-figures/low_power_modes.png}
    
    \caption{Graf znázorňujíci dynamiku spotřeby mikrokontroléru v průběhu času při využití aktivního a nízkoenergetického režimu}
    \label{fig:low-power-modes}
\end{figure}

Ačkoliv nízkoenergetické režimy přinášejí značné úspory energie a jsou nezbytné pro zařízení napájená z baterií, u dataloggerů s velkým objemem zaznamenávaných dat mohou představovat 
významná omezení. Tyto režimy sice snižují energetickou náročnost systému, avšak zároveň omezují schopnost mikrokontroléru rychle reagovat na události. Spánkové stavy, které minimalizují 
spotřebu energie, často vedou k delšímu zpoždění při probuzení a nižší dostupnosti kritických periferií. V aplikacích, kde je vyžadována okamžitá odezva na externí podněty nebo nepřetržité 
zpracování velkého množství dat, může tento faktor negativně ovlivnit spolehlivost a efektivitu záznamníku. \cite{embedded_low_power_modes}

V těchto případech je proto vhodné zvážit provozní podmínky a očekávanou dostupnost systému. Pokud záznamník pracuje s velkým datovým tokem a má možnost být po dobu záznamu stále dostupné 
externí napájení, může být výhodnější upustit od implementace nízkoenergetických režimů a místo toho optimalizovat architekturu systému pro nepřetržitý provoz s důrazem na výkon a rychlou 
odezvu. \cite{analog_devices_low_power_modes}

\section{Typy médií pro záznam dat}
Hard disk, vyhody nevyhody,... 

\chapter{Návrh digitálního záznamníku}

\section{Existujících řešení digitálních záznamníku}

\section{Výběr vhodné platformy}
Neni vhodne implementovat zaznamnik na pocitacovem systemu, je treba zvolit stragii implementace na mikroradici, porovnat FRDM-MCXN947, Arduino, Raspberry
\subsection{FRDM-MCXN947}

\subsection{Arduino}

\subsection{Linux based - Raspberry}

Pripraven i popis jadra ARM Cortex-M33, ktere vyuziva FRDM-MCXN947.

% ----------------------------------------------------
% DALSI NAZVY: Volba datového úložiště, Výběr externího uložiště pro záznam dat, Možnosti způsobu ukládání získaných dat
\section{Přístupy k ovládaní úložiště}
Obecný popis, proč je potřeba externí uložiště, že by se data mohla ukládat i v RAM paměti, ale že by tam moc dlouho nevydržela, 

\subsection{SDHC}

\subsection{SPI}
\subsection{Quad-SPI flash}


% ----------------------------------------------------
% DALSI NAZVY: Volba datového úložiště, Výběr externího uložiště pro záznam dat
\section{Možnosti správy dat - souborové systémy} 
Obecný popis, souborový systém je zodpovědný za organizaci, správu a přístup k datům na zvoleném úložném médiu.

\subsection{FATFS}

\subsection{Chan FATFD}

\subsection{LittleFS}


% ----------------------------------------------------

\section{Výběr řízení přístupu k získaným datům}

\subsection{USB Mass Storage}

\subsection{Media Transfer Protocol}

\subsection{Human Interface Device}

% ----------------------------------------------------
\section{Výběr zdroje času}

\subsection{Obvod reálného času}

\subsection{Interní časovač}

\subsection{Bezdrátová komunikace (GPS/NTP)}

% ----------------------------------------------------
\section{Výber přístupu řízení běhu aplikace}
Srovnani obecne bare-metal a RTOS.


\subsection{Bare-Metal}

% TODO: Zde to asi spis rozdelit na Baremetall vs. RTOS a pak uvest priklady jako FreeRTOS a ZephyrRTOS
\subsection{RTOS}
\subsubsection{FreeRTOS}
Konkretní výhody a nevýhody FreeRTOS

\subsubsection{ZephyrRTOS}
Konkretní výhody a nevýhody ZephyrRTOS

% ----------------------------------------------------

\section{Architektura systému digitálního záznamníku}
Popis architektury na základě vybraných komponent. Popis blokového diagramu.

% ----------------------------------------------------

\section{Volitelné rozšíření}
\subsection{Měření teploty}

% ----------------------------------------------------

\subsection{Řešení problému synchronizace času}

% ----------------------------------------------------

\chapter{Realizace hardwaru}
Popis co všechno je již na platformě FRDM MCXN947, z toho ti vyplyne co všechno bude ještě muset nabízet expanzní deska.
\section{Základová deska}
\section{Expanzní deska}

\section{Mechanická část}
Jak jsem měřil spotřebu, jak jsem spočítal hodnoty kondenzátorů

\chapter{Softwárová implementace}

\section{Záznamové vlákno}

\section{USB Mass Storage vlákno}

\section{Signalizace stavu systému}

% ==================================================== 
\chapter{Testování systému}
Obecne proc je potreba testovat, jake jsou moznost testovani a validace vestavenych systemu

% ----------------------------------------------------

\section{Testování a validace}

\subsection{Funkcionální testování}
Popis skriptů pro automatické testování, popis výsledků, atd, ...

\subsection{Kontrola bezpečnosti kódu}
MISRA

% ----------------------------------------------------

\section{Limitace systému}
\label{limitace}

% ----------------------------------------------------

\section{Možná rozšíření záznamníku}
\label{mozne_rozsireni}

\chapter{Závěr}
\label{zaverPrace}


%===============================================================================

% Pro kompilaci po částech (viz projekt.tex) nutno odkomentovat
%\end{document}
